% !TEX root = ../Thesis.tex
\chapter{PolySQL Syntax - Freshness Extension}
\todo{Maybe also add MQL}
This chapter provides an in-depth extension to the existing PolySQL Syntax for freshness related queries.
The original PolySQL Syntax will not be illustrated in this chapter.

All valid extensions for Freshness must consequently begin with the keywords \emph{WITH FRESHNESS}.
They are attached as an optional leaf expression for every \emph{SELECT} statement.

\tocless\section{PolySQL}


\begin{lstlisting}[language=sql]
    SELECT * FROM tableName 
    [ WITH FRESHNESS 
        [ 
            ( 
                TIMESTAMP 
                | 
                <DELAY> 
                | 
                <INDEX> 
            )   
        ]];
\end{lstlisting}

\tocless\subsection{Absolute Timestamp}

\begin{verbatim}
    SELECT * FROM dummy WITH FRESHNESS TIMESTAMP '2022-07-04 06:30';
\end{verbatim}


\tocless\subsection{Relative Timestamp - Absolute Delay}

\begin{verbatim}
    SELECT * FROM dummy WITH FRESHNESS 3 SECOND ABSOLUTE;
\end{verbatim}

\begin{verbatim}
    SELECT * FROM dummy WITH FRESHNESS 3 HOUR ABSOLUTE;
\end{verbatim}

\begin{verbatim}
    SELECT * FROM dummy WITH FRESHNESS 3 MINUTEs ABSOLUTE;
\end{verbatim}

\tocless\subsection{Relative Delay}

\begin{verbatim}
    SELECT * FROM dummy WITH FRESHNESS 3 SECOND DELAY;
\end{verbatim}

\begin{verbatim}
    SELECT * FROM dummy WITH FRESHNESS 3 HOUR DELAY;
\end{verbatim}

\begin{verbatim}
    SELECT * FROM dummy WITH FRESHNESS 3 MINUTEs DELAY;
\end{verbatim}



\tocless\subsection{Freshness Index}

\begin{verbatim}
    SELECT * FROM dummy WITH FRESHNESS 0.6;
\end{verbatim}

\begin{verbatim}
    SELECT * FROM dummy WITH FRESHNESS 60%;
\end{verbatim}



\tocless\subsection{Refresh Operations}

\begin{verbatim}
    ALTER TABLE dummy REFRESH ALL PLACEMENTS;
\end{verbatim}

\begin{verbatim}
    ALTER TABLE dummy REFRESH ALL PLACEMENTS ON STORE storeName;
\end{verbatim}
