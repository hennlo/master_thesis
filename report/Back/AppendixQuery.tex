% !TEX root = ../Thesis.tex
\chapter{PolySQL Syntax - Freshness Extension}


This chapter provides an extension to the existing PolySQL Syntax, necessary to work with freshness.
The original PolySQL\footnote{https://polypheny.org/documentation/PolySQL/} Syntax will not be illustrated in this chapter.

\tocless\section{Freshness Specification}
\label{a:syntax}

All valid extensions for freshness-related queries must consequently begin with the keywords \emph{WITH FRESHNESS}.
They are attached as an optional leaf expression for every \emph{SELECT} statement.

\tocless\subsection{Absolute Timestamp}

\begin{lstlisting}[language=sql]
    SELECT * FROM tableName WITH FRESHNESS TIMESTAMP '2022-07-04 06:30';
\end{lstlisting}
\vspace*{-5mm}
\tocless\subsection{Relative Timestamp - Absolute Delay}

\begin{lstlisting}[language=sql]
    SELECT * FROM tableName WITH FRESHNESS 3 SECOND ABSOLUTE;

    SELECT * FROM tableName WITH FRESHNESS 3 HOUR ABSOLUTE;

    SELECT * FROM tableName WITH FRESHNESS 3 MINUTE ABSOLUTE;
\end{lstlisting}
\tocless\subsection{Relative Delay}
\begin{lstlisting}[language=sql]
    SELECT * FROM tableName WITH FRESHNESS 3 SECOND RELATIVE;

    SELECT * FROM tableName WITH FRESHNESS 3 HOUR RELATIVE;

    SELECT * FROM tableName WITH FRESHNESS 3 MINUTE RELATIVE;
\end{lstlisting}
\tocless\subsection{Freshness Index}
\begin{lstlisting}[language=sql]
    SELECT * FROM tableName WITH FRESHNESS 0.6;
\end{lstlisting}
\tocless\section{Replication Constraints}
In order to use freshness-related queries, the system needs to be consequently altered.
The following sections illustrate how to transform individual Data Placements.
\tocless\subsection{Adapting the Replication Strategy}
\begin{lstlisting}[language=sql]
    ALTER TABLE tableName MODIFY PLACEMENT ON STORE storeName WITH REPLICATION EAGER;

    ALTER TABLE tableName MODIFY PLACEMENT ON STORE storeName WITH REPLICATION LAZY;
\end{lstlisting}
\tocless\subsection{Adapting the Placement State}
\begin{lstlisting}[language=sql]
    ALTER TABLE tableName MODIFY PLACEMENT ON STORE storeName WITH STATE REFRESHABLE;

    ALTER TABLE tableName MODIFY PLACEMENT ON STORE storeName WITH STATE OUTDATED;
\end{lstlisting}
\tocless\subsection{Refresh Operations}
\begin{lstlisting}[language=sql]
    ALTER TABLE tableName REFRESH ALL PLACEMENTS;

    ALTER TABLE tableName REFRESH ALL PLACEMENTS ON STORE storeName;
\end{lstlisting}
