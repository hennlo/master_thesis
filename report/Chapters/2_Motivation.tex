% !TEX root = ../Thesis.tex
\chapter{Motivational Scenario}
\label{c:motivation}

Explain scenario and why especially necessary on polystore systems
transactional UPTODATE workload on postgres and 
Analytical in in-memory order column-oriented DB. 

(Imagine a distributed system)

COnsider locking in a distributed setup. Why this is limited by the slowest performing node.
(maybe cite) Amdahls law .

\todo{This might not be suitable in Motivational Scenario}
We can now harvest the benefits of a polystore system with their distinct replication engines

Imagine an \emph{Enterprise Resource Planning} (ERP) system, containing several different customers.
This comprehensive system is able to provide purchasing, accounting, inventory stocking and warehouse management within one system.
Since these systems are inherently used to reproduce an entire supply chain, they can be easily be used to report on different sales metrics
or run analytical queries to aggregate certain multidimensional results. This is certainly desirable to analyze the current state of the company as well as adequately adapt 
to new trends or opportunities. However, even in we imagine that we have a distributed system \\
If our system however only supports strong consistency by eagerly replicating all incoming updates to every available node, 
we are limited by the slowest performing node in this setup. During these write operations no could

Explain that this system in this scenario does not allow concurrent writes and reads. Since reads do not alter that data, we can provide multiple 




However, since ERP \todoMissing{Cite this} are strong transactional systems receiving write-heavy workload the write speed is essentially bound by the lowest performing

This does not only limit locking situations but also increases the databases' response time. Furthermore, this enables this system even in a distributed setup to work 
with several nodes. Even if not all nodes are directly notifying the system that they have succesfully finished the operation we could continue with the operation.
Depending on the tolerated level of consistency we give in on consistency to improve availability. This can of course also be tuned to wait until a majority or defined number of nodes 
have committed the transaction.

\todoMissing{Imagine a global manufacturing company that is responsible for the entire supply chain. enables direct purchase through an ERP system which has several thousand
customers, production-ready assembly lines  }
\todoMissing{ Within the company, each devision has its own core functionality and although data is important within all those systems.}

All activity is primarly executed through a central ERP system. 

Since this company has locations around the world the ERP system has an evenly distributed load throughout the day without any peak performance windows.
The availability as well as the processing throughput of this system is crucial for the company to stay competitive.

While the processing department concerned with sales and production, the business intelligence (BI) team on the other hand needs to analyze to provide sales forecasts recognize 
upcoming trends. Since the company and data is still globally distributed and each entity additionally stores some information locally, that are unique to this branch or devision. 
So any analysis done by the BI team needs to retrieve data from different locations maybe even companies. Although, these
They cannot interfere with the daily-operation of the company. 

Although this is provided within one central system the use cases and requirements are fundamentally different.
Since the company is globally distributed among several locations the data storage engines are distributed across the globe.
Since this is a central system the availability as well as the consistency is important, to protect itself against failures while still providing an acceptable throughput.


For the sake of simplicity any legal constraints on archiving financial data are omitted. 

\section{Goal?}
The system therefore needs to be able to allow transactional as well as analytical workload in parallel. 