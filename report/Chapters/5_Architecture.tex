% !TEX root = ../Thesis.tex
\chapter{System Architecture}
\label{c:architecture}

In this chapter we briefly describe and illustrate a simplified version of Polyphenys current architecture.\\
This extends the foundations laid out in Chapter \ref{c:Foundation} and sets them in context of the existing system model.

\todoMissing{Polypheny multi-model teherfore tables are considered entities }


%%%%%%%%%%%%%%%%%%%%%%%%%%%%%%%%%%%%%%%%%%%%%%%%%%%%%%%%%%%%%%%%%%



\section{Polypheny-DB}
PolyDBMS \cite{polypheny2021}
\todo{Add PolyDBMS cite Love Marriage or Marriage of convenience}

\textit{Polypheny-DB} is an Open-Source project\footnote{https://polypheny.org/} developed by 
the \textit{Database and Information Systems} (DBIS) group of the University of Basel.\\

Polypheny-DB is a self-adaptive polystore that provides cost- and workload aware access to heterogeneous data\cite{poly2020}.

Compared to other systems like \textit{C-Store}\cite{cstore_2005} or \textit{SAP HANA} \cite{hana_2012}, 
Polypheny-DB does not provide its own set of different storage engines to support 
different workload demands.\\
Instead, it acts as a higher-order DBMS which provides a single-point of entry to 
a variety of possible databases like 
\textit{Cassandra}\footnote{https://cassandra.apache.org/}, 
\textit{PostgreSQL}\footnote{https://www.postgresql.org/} 
and \textit{MonetDB}\footnote{https://www.monetdb.org/}. 
These can be integrated, attached and managed by Polypheny-DB which will incorporate the underlying 
heterogenous data storage engines with their different data structures. It is 
desigend to abstract applications from the physical execution engine while profiting from 
performance improvements through cross-engine executions. 
\\
For incoming queries Polypheny-DB's routing engine will automatically analyze the query and decide 
which store will provide the best response. The query is then explicitly routed to these data stores. 
This approach can be characterized as a dynamically optimizing data management layer for different workloads.


\subsection{Data Placements}
\subsubsection{Vertical Placements} 
\subsubsection{Data Placements}

\subsection{Query Routing}

\subsection{Concurrency Control}


