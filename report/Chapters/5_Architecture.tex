% !TEX root = ../Thesis.tex
\chapter{System Architecture}
\label{c:architecture}

The implementation and concept is based on the polystore system Polypheny-DB\footnote{https://github.com/polypheny/Polypheny-DB}.
In this chapter we briefly describe and illustrate a simplified version of Polypheny-DBs current architecture.\\
This extends the foundations laid out in Chapter \ref{c:Foundation} and sets them in context of the existing system model.

\todoMissing{Polypheny multi-model therfore tables are considered entities }


%%%%%%%%%%%%%%%%%%%%%%%%%%%%%%%%%%%%%%%%%%%%%%%%%%%%%%%%%%%%%%%%%%



\section{Polypheny-DB}
PolyDBMS \cite{polypheny2021}
\todo{Add PolyDBMS cite Love Marriage or Marriage of convenience}

\textit{Polypheny-DB} is an Open-Source project\footnote{https://polypheny.org/} developed by 
the \textit{Database and Information Systems} (DBIS) group of the University of Basel.\\

Polypheny-DB is a self-adaptive polystore that provides cost- and workload aware access to heterogeneous data\cite{poly2020}.

Compared to other systems like \textit{C-Store}\cite{cstore_2005} or \textit{SAP HANA} \cite{hana_2012}, 
Polypheny-DB does not provide its own set of different storage engines to support 
different workload demands.\\
Instead, it acts as a higher-order DBMS which provides a single-point of entry to 
a variety of possible databases like 
\textit{MongoDB}\footnote{https://www.mongodb.com/}, 
\textit{Neo4j}\footnote{https://neo4j.com/},
\textit{PostgreSQL}\footnote{https://www.postgresql.org/} 
and \textit{MonetDB}\footnote{https://www.monetdb.org/}. 
These can be integrated, attached and managed by Polypheny-DB which will incorporate the underlying 
heterogenous data storage engines with their different data structures. 
It is desigend to abstract applications from the physical execution engine while profiting from 
performance improvements through cross-engine executions. 
\\
For incoming queries Polypheny-DB's routing engine will automatically analyze the query and decide 
which store will provide the best response. The query is then explicitly routed to these data stores. 
This approach can be characterized as a dynamically optimizing data management layer for different workloads.

\todoMissing{polypheny support multi-model databsaes for relational, document, graph in memroy ...}

%%%%%%%%%%%%%%%%%%%%%%%%%%%%%%%%%%%%%%%%%%%%%%%%%%%%%%%%%%%%%%%%%%

\section{Placements}
Placements are considered to be Polyphenys virtual representation of physical entities.
They act as an abstraction between the polystore layer and the physical representation of an entity. 
Mostly used within the PolyDBMS itself they help to assist the logical routing process of Polypheny-DB.



\subsection{Data Placements}

A Data Placement is essentially a virtual representation of the physical entity residing on a given store.
A store in Polypheny is an underlying physical data storage which is 
attached to Polypheny-DB.
All attached stores can be used to hold several fragments of data. 
During routing decisions stores are automatically taken into consideration if they are designated for the associated data 

It contains information on available columns ($\rightarrow$ Column Placements), partitions ($\rightarrow$ Partition Placements)
as well as properties unique to this store.

A table can therefore contain several Data Placements with different capabilities and properties. \todoMissing{Image}

are used along the idea of Column Placement. A Data Placement 
    is a representation of table with all placed columns on a specific physical store.

    When a table is created on Polypheny-DB it is an ordinary structure placed onto 
    one store. Such a table consists of one to \textit{n}-columns.
    In the context of vertical partitioning a subset of these \textit{n}-columns can now 
    be placed onto another store in form of a \textit{Data Placement}.
    This can either be done by evenly distributing the columns onto these stores 
    or by simply replicating the subset to the second store.\\

    


\subsection{Column Placements} 
\todoMissing{Image}
are needed to fulfill the intended flexibility of Polypheny-DB. 
    Column Placements are instances of a column placed on a specific store.
    These placements are the result of the extended vertical partitoning of a table.
    
Column Placements are instances of a column placed on a specific store.
These placements are the result of the extended vertical partitioning of a table.
They are considered unique per column on a cluster.

\todoMissing{Maybe summarize this under Data Placement}
As already discussed in \ref{sec:part}, vertical partitioning refers to the logical 
separation of the data structure by columns to obtain logically connected objects throughout 
the database. 
Polypheny-DB extends this functionality to vertically partition tables
column wise, which allows a table itself to be split further into a disjoint 
set of columns. This extension provides the functionality to place columns 
rather freely on a store without replicating the complete table. 
Although these columns are logically bound to a table there is no need 
to replicate the complete structure to a desired store. In some cases 
this does not only result in an optimized access of the data part but 
also saves data overhead on the specific store.\\
This functionality enables Polypheny-DB to adapt the data structure to continuously 
varying use cases.\\



\subsection{Partition Placements}
Due to the partition function NONE every table entity inside Polypheny-DB is considered to be partitioned.  Hence consisting only of one partition.
Additioanlly, Polypheny suppports the most common partition algorithms like HASH (), range or list(). 

A Partition Placement is 



%%%%%%%%%%%%%%%%%%%%%%%%%%%%%%%%%%%%%%%%%%%%%%%%%%%%%%%%%%%%%%%%%%

\section{Catalog}


\subsection{Query Routing}
\todoMissing{Image}
Since every query has to go through the abstraction layer to guarantee correctness 
and consistency, Polypheny-DB can consult the systems \textit{Catalog} to retrieve the
location of all relevant data. This is done by gathering all 
\textit{Column placements} needed by the query.\\ 
If the requested data indeed happens to be distributed
on severall stores. The central routing engine will join all relevant and distinct 
placements to construct the result set. Hence, the query is always routed to stores which 
hold relevant data.


Since partitions are mere logical identifiers there main usage is to locate data fragments or the location where a query should route a statement to.
One physical table can therefore hold several partitions.
During routing if a partition has been identified it is checked for every store involved whether it contains and associated partition placement.
Although, this routing method is quite fast there are several problems concerning the separability of data.
This imposes for one the difficulty to retrieve data belonging to exactly one partition out of a table which contains several partitions and secondly difficult 
to migrate data from individual partitions to another store.\\
Since it is rather complex to extract all relevant partitions needed for a specific query especially when combining vertical and horizontal partitioning
the concept of \textit{Worst case Routing} was introduced.\\
This routing mechanism aims to improve performance, when the process of identifying the correct partitions for a query would be too complicated and 
therefore also reduce the overall performance.   
This is the reason why currently, a \textit{Full Placement} has to be enforced for all three Partition Managers to support the functionality of worst case routing.
A Full Placement in that sense refers to a placement on a store which contains all partitions of a specific table and can therefore be used as a fallback scenario.

However, due to the adjustments to partitioning including the new Partition Placements which are represented by their own individual physical tables, the constraint imposed by logical 
partitions and therefore the necessity of a full placement per table can be removed.\\
Queries are  now able to flexibly combine vertical and horizontal partitioning to truly leverage the power of Polypheny-DB.
The routing for SELECT-queries is now simplified since it aims to find for each partition all requested columns by 
joining Column Placements per partition first and then applying a UNION over all accessed partitions to build the required result set.


\section{Workload monitoring}

The workload monitoring is already partially existing and supplies Polypheny-DB with the capability to extract useful information out of several events occurring in the system.
This information can include runtime characteristics, execution time, accessed columns, returned number of rows and the timestamp when the event trigger was recorded.
These attributes are attached to one event and used to create and expose individual and topic-oriented data points to be retrieved later on for dedicated analysis.\\

The service itself is running entirely in the background of the system observing and tracking new events and placing them into a FIFO queue to be processed asynchronously.
For one, this decoupling of event creation and processing reduces the workload of the overall system, since background jobs won't have to wait for a synchronous response, 
moreover it provides Polypheny-DB the possibility to deploy additional workers which can concurrently process the queue in case of a high system load when more events are generated 
that one worker can handle alone. \\
At a fixed time interval the queue is processed by \textit{Monitoring Queue Workers} which take a handful of elements from the queue and analyze them in terms of attached information 
and ultimately creates several \textit{Data Points}.
These Data Points can vary from \textit{Query Events}, \textit{DML Events} or other use case specific triggers which are worth tracking.
Once these Data Points have been created they will be persistently written to disk to ensure recoverability and can be used to enrich internal analysis of events that have 
happened in the past.

\subsection{Measuring Access Rates}
The current functionality of the \textit{Workload Monitor} is already capable of observing events in the system.
To fulfil the requirements for our use case the information for statement specific events have been enriched together with the executed statement type 
a list of all accessed partitions have been attached to the monitoring object.\\
Since the monitoring system currently only supports DMLs and queries one of those events is converted into one or several data points to be retrieved later on. 
This is sufficient for this use case since we are only interested in exactly those types of statements to characterize if the access was a read, write or any of those when trying to 
find the total access rate.\\


But not only the frequency of access but also the recency of access should be considered therefore timeframes have been introduced to supplement the functionality.
Since the monitoring service lets you retrieve information within a certain timeframe we can directly fulfil the temperature requirement to retrieve
recently and frequently accessed partitions. 


\subsection{Concurrency Control}

Given Polyphenys current architecture all incoming queries has to be delivered through the poly-layer, acting as a central instance.
Since we assume that there is no direct interaction with the underlying systems there is no immediate risk of inconsistencies. 
This allows the utilization of SS2PL to handle concurrency control only within Polypheny-DB for correct isolation treatment.

