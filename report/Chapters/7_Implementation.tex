% !TEX root = ../Thesis.tex
\chapter{Implementation}
\label{c:implementation}

This chapter describes an implementation in correspondence to the concepts elaborated in chapter \ref{c:concept}. 
These concepts are applied to Polypheny-DB, a particular polystore system.\\
First the current architecture and all relevant components and modules of this system are described as a foundation to put the implementations in place ($\rightarrow section \ref{sec:architecture}$).

After this brief introduction the corresponding building blocks, necessary to establish the concept of freshness-awareness are described.

Change Data Capture ($\rightarrow section \ref{sec:}$)
Replication Algorithm($\rightarrow section \ref{sec:}$)
Freshness specifications($\rightarrow section \ref{sec:}$)
Freshness evaluation($\rightarrow section \ref{sec:}$)
Freshness constraints($\rightarrow section \ref{sec:}$)


\todo{Talk about each implementation step for the building blocks in general, but if there are deviations e.g. in languages briefly differentiate them with bullet points
and add them to the appendix }








%%%%%%%%%%%%%%%%%%%%%%%%%%%%%%%%%%%%%%%%%%%%%%%%%%%%%%%%%%%%%%%%%%



\section{Polypheny-DB}
\label{sec:architecture}


The implementation is based on the polystore system Polypheny-DB\footnote{https://github.com/polypheny/Polypheny-DB}.
In this chapter we briefly describe and illustrate a simplified version of Polypheny-DBs current architecture.\\
This extends the foundations laid out in Chapter \ref{c:Foundation} and sets them in context of the existing system model.


PolyDBMS \cite{polypheny2021}
\todo{Add PolyDBMS cite Love Marriage or Marriage of convenience}

\textit{Polypheny-DB} is an Open-Source project\footnote{https://polypheny.org/} developed by 
the \textit{Database and Information Systems} (DBIS) group of the University of Basel.\\

Polypheny-DB is a self-adaptive polystore that provides cost- and workload aware access to heterogeneous data\cite{poly2020}.

Compared to other systems like \textit{C-Store}\cite{cstore_2005} or \textit{SAP HANA} \cite{hana_2012}, 
Polypheny-DB does not provide its own set of different storage engines to support 
different workload demands.\\
Instead, it acts as a higher-order DBMS which provides a single-point of entry to 
a variety of possible databases like 
\textit{MongoDB}\footnote{https://www.mongodb.com/}, 
\textit{Neo4j}\footnote{https://neo4j.com/},
\textit{PostgreSQL}\footnote{https://www.postgresql.org/} 
and \textit{MonetDB}\footnote{https://www.monetdb.org/}. 
These can be integrated, attached and managed by Polypheny-DB which will incorporate the underlying 
heterogenous data storage engines with their different data structures. 
It is desigend to abstract applications from the physical execution engine while profiting from 
performance improvements through cross-engine executions. 
\\
For incoming queries Polypheny-DB's routing engine will automatically analyze the query and decide 
which store will provide the best response. The query is then explicitly routed to these data stores. 
This approach can be characterized as a dynamically optimizing data management layer for different workloads.\\
Due to its inherent architecture and the possibility to replicate data across different homogenous as well as heterogeneous stores, it is also able to cluster, specific stores 
on a table entity level, although the underlying stores might not support this natively. 
This leverages Polypheny-DB to a data orchestration platform. 



\todoMissing{polypheny support multi-model databsaes for relational, document, graph in memroy ...}

%%%%%%%%%%%%%%%%%%%%%%%%%%%%%%%%%%%%%%%%%%%%%%%%%%%%%%%%%%%%%%%%%%

\subsection{Placements}
Placements are considered to be Polyphenys virtual representation of physical entities.
They act as an abstraction between the polystore layer and the physical representation of an entity. 
Mostly used within the PolyDBMS itself they help to assist the logical routing process of Polypheny-DB.


\begin{description}
    \item [Data Placements] A Data Placement is essentially a virtual representation of the physical entity residing on a given store.
    A store in Polypheny is an underlying physical data storage which is 
    attached to Polypheny-DB.
    All attached stores can be used to hold several fragments of data. 
    During routing decisions stores are automatically taken into consideration if they are designated for the associated data 
    
    It contains information on available columns ($\rightarrow$ Column Placements), partitions ($\rightarrow$ Partition Placements)
    as well as properties unique to this store.
    
    A table can therefore contain several Data Placements with different capabilities and properties. \todoMissing{Image}
    
    are used along the idea of Column Placement. A Data Placement 
        is a representation of table with all placed columns on a specific physical store.
    
        When a table is created on Polypheny-DB it is an ordinary structure placed onto 
        one store. Such a table consists of one to \textit{n}-columns.
        In the context of vertical partitioning a subset of these \textit{n}-columns can now 
        be placed onto another store in form of a \textit{Data Placement}.
        This can either be done by evenly distributing the columns onto these stores 
        or by simply replicating the subset to the second store.\\

    \item [Column Placements] \todoMissing{Image}
    are needed to fulfill the intended flexibility of Polypheny-DB. 
        Column Placements are instances of a column placed on a specific store.
        These placements are the result of the extended vertical partitoning of a table.
        
    Column Placements are instances of a column placed on a specific store.
    These placements are the result of the extended vertical partitioning of a table.
    They are considered unique per column on a cluster.
    
    \todoMissing{Maybe summarize this under Data Placement}
    As already discussed in \ref{sec:part}, vertical partitioning refers to the logical 
    separation of the data structure by columns to obtain logically connected objects throughout 
    the database. 
    Polypheny-DB extends this functionality to vertically partition tables
    column wise, which allows a table itself to be split further into a disjoint 
    set of columns. This extension provides the functionality to place columns 
    rather freely on a store without replicating the complete table. 
    Although these columns are logically bound to a table there is no need 
    to replicate the complete structure to a desired store. In some cases 
    this does not only result in an optimized access of the data part but 
    also saves data overhead on the specific store.\\
    This functionality enables Polypheny-DB to adapt the data structure to continuously 
    varying use cases.\\

    \item [Partition Placements] Due to the partition function NONE every table entity inside Polypheny-DB is considered to be partitioned.  Hence consisting only of one partition.
    Additioanlly, Polypheny suppports the most common partition algorithms like HASH (), range or list(). 
    
    A Partition Placement is 
\end{description}





%%%%%%%%%%%%%%%%%%%%%%%%%%%%%%%%%%%%%%%%%%%%%%%%%%%%%%%%%%%%%%%%%%

\subsection{Catalog}


\subsection{Query Routing}

Routing or cache plan optimizat execute

\todoMissing{Image}
Since every query has to go through the abstraction layer to guarantee correctness 
and consistency, Polypheny-DB can consult the systems \textit{Catalog} to retrieve the
location of all relevant data. This is done by gathering all 
\textit{Column placements} needed by the query.\\ 
If the requested data indeed happens to be distributed
on several stores. The central routing engine will join all relevant and distinct 
placements to construct the result set. Hence, the query is always routed to stores which 
hold relevant data.


Since partitions are mere logical identifiers there main usage is to locate data fragments or the location where a query should route a statement to.
One physical table can therefore hold several partitions.
During routing if a partition has been identified it is checked for every store involved whether it contains and associated partition placement.
Although, this routing method is quite fast there are several problems concerning the separability of data.
This imposes for one the difficulty to retrieve data belonging to exactly one partition out of a table which contains several partitions and secondly difficult 
to migrate data from individual partitions to another store.\\
Since it is rather complex to extract all relevant partitions needed for a specific query especially when combining vertical and horizontal partitioning
the concept of \textit{Worst case Routing} was introduced.\\
This routing mechanism aims to improve performance, when the process of identifying the correct partitions for a query would be too complicated and 
therefore also reduce the overall performance.   
This is the reason why currently, a \textit{Full Placement} has to be enforced for all three Partition Managers to support the functionality of worst case routing.
A Full Placement in that sense refers to a placement on a store which contains all partitions of a specific table and can therefore be used as a fallback scenario.

However, due to the adjustments to partitioning including the new Partition Placements which are represented by their own individual physical tables, the constraint imposed by logical 
partitions and therefore the necessity of a full placement per table can be removed.\\
Queries are  now able to flexibly combine vertical and horizontal partitioning to truly leverage the power of Polypheny-DB.
The routing for SELECT-queries is now simplified since it aims to find for each partition all requested columns by 
joining Column Placements per partition first and then applying a UNION over all accessed partitions to build the required result set.




\subsection{Concurrency Control}

Given Polyphenys current architecture all incoming queries has to be delivered through the poly-layer, acting as a central instance.
Since we assume that there is no direct interaction with the underlying systems there is no immediate risk of inconsistencies. 
This allows the utilization of SS2PL to handle concurrency control only within Polypheny-DB for correct isolation treatment.



%%%%%%%%%%%%%%%%%%%%%%%%%%%%%%%%%%%%%%%%%%%%%%%%%%%%%%%%%%%%%%%%%%
\section{Multiple Versions}

as described in Section \ref{sec:part}, since read-only queries typically benefit directly from data partitioning, they are suitable candidates to base our freshness awareness on.
Therefore, we propose to define outdatedness on the state of a specific partition placement.
Although the entire data placement, could be labeled as outdated or rather receive updates lazily, some of 
these partitions could already be up-to-date again, while others still remain outdated.



%%%%%%%%%%%%%%%%%%%%%%%%%%%%%%%%%%%%%%%%%%%%%%%%%%%%%%%%%%%%%%%%%%

\section{Locking - Isolation Level}




%%%%%%%%%%%%%%%%%%%%%%%%%%%%%%%%%%%%%%%%%%%%%%%%%%%%%%%%%%%%%%%%%%


\section{Lazy Replication}

\subsection{Replication Strategy}
Can be used to selectively define which data placement shall be updated lazily.
\todoMissing{Statement} 
\begin{verbatim}
    ALTER TABLE tableName MODIFY PLACEMENT ON STORE storeName WITH REPLICATION LAZY;
\end{verbatim}

This replication strategy is added as part of the newly added data placement properties. Since Data Placements inherently carry the information, what column and partitions reside
on a given store, they were extended to now also hold information on data placement specific properties.


\subsection{MultipleVersions}
\todoMissing{Link to concept}
As the name suggests a multi-version database would be ideal and the obvious choice for such an approach.
These databases will automatically generate a new version for a data object on each modification.
Due to their properties we would immediately have the information on the validity-interval of the version, its update time as well as older and newer versions.
This would directly allow us to utilize these versions on freshness-related queries. But, it would also imply the utilization of MVCC.
However, Polypheny-DB currently only supports SS2PL for its concurrency control. 

Additionally, there are a lot of trade-offs to consider when using MVCC as previously stated in \ref{sec:concurrency_control}.
For one, it has a large data footprint which is already inherently larger in a polystore system, given the redundant data storage.
Furthermore, since all transactions are handled within the polystore layer we need a correct and serializable execution order on all stores \todoMissing{Remove Correctness Guarantees}. 
Hence, we need to provide a more rigorous and uniform schedule on all stores, which is not directly provided by MVCC.
That is why we stick with SS2PL and refrain from using the automatic versioning provided by a multi-version database.\\



\subsection{Replication State}

Since the replication strategy is bound to an independent Data Placement we still need the possibility to define how the actual Partition Placements, that hold the data 


\begin{description}
    \item [UPTODATE]
    \item [REFRESHABLE]
    \item [INFINITELY-OUTDATED]
\end{description}

Furthermore each Partition Placement is enriched with the most recent update information to support various freshness metrics.
This update information contains insights such as the id of the last comitted update transaction, the corresponding commit and update timestamp as well as the number of 
modifications that 


\subsection{Replication Engine}
Core functionality that transforms capture objects into distinct replication Objects and pipes them to specific execution engines.


\subsection{Automatic Replication Algorithm}

\todoMissing{In regards to CDC, if we observe that the number of pending update operations exceed a certain threshold for example 50\% of the total number of modifications of the master
we can directly remove all pending updates and execute a primary snapshot copy because this is faster then reexecuting the operation again. }

\todoMissing{Since the replication or refresh propagation is done operation-wise we actually loosened the heavy SS2PL 
constraints that we require on the primary updates. Since we already have a serializable schedule after execution we also 
    know per entity and each partition placement the correct execution order that we have to apply the data changes on. So it is not necessary to free the resources 
    after the entire transaction has been replicated but right after each operation. However since Polypheny-DB currently only supports a SS2PL approach we can mimic this behaviour by scheduling refresh transaction contianing only one update or even a subset of the operations.}

\todoMissing{Explain potential optimization steps  that we can analyze the queue and can aggregate certain steps or avoid 
certain operations if we execute batch wise and one UPDATE operation e.g. updates the same primary key}

\subsection{Manual Refresh Operations}
\todoMissing{Not specifically in manual, altering or inhernetly modifying a data placement with DDLs leaves them as INFINITELY OUTDATED }
\todoMissing{ Or is this automatically detected since we centrally store the replication object and can apply it to arbitrary placements disregarding their local columns or partitions.}

\todoMissing{LazyWorkerThreads - were each worker is responsbile for carrying out one replication or requeing it after it has failed}



\subsection{Refresh strategies}
As already stated above, there are essentially three ways to update a secondary node. 
For the automatic execution of propagation transactions a periodic execution has several downsides hence we propose to schedule
refreshment on the basis of the load on the outdated placements. Therefore, we will extend the adapter of the underlying stores to gather metrics on current workload
average response times in order to enable the poly-layer to decide based on this metrics when it is suitable to carry out a propagation-transaction.


Since it might be possible that a user may consider refreshing a placement manually, we have proposed refresh transactions. With this a user has the possibility to refresh
any replicas which are classified as outdated to a specific freshness level.
\begin{verbatim}
ALTER TABLE dummy REFRESH PLACEMENT ON STORE outdated_store 
UNTIL <TIMESTAMP>;
\end{verbatim}
Without any specification the placements shall be updated to the most recent state of the up-to-date version:
\begin{verbatim}
ALTER TABLE dummy REFRESH PLACEMENT ON STORE outdated_store;
\end{verbatim}

For this operation only freshness levels greater than the current local freshness level can be considered.
Furthermore, this should also provide the possibility to refresh all placements of an object.
\begin{verbatim}
ALTER TABLE dummy REFRESH ALL PLACEMENTS;
\end{verbatim}

Although such refresh-transactions can be executed on any placement, it will have no effect on primaries and simply omit their execution. 


%%%%%%%%%%%%%%%%%%%%%%%%%%%%%%%%%%%%%%%%%%%%%%%%%%%%%%%%%%%%%%%%%%

\section{Freshness Query}
This can be achieved by extending the query functionality of Polypheny-DB.

Moreover, to improve the user experience the query tree paths should be extended to specify which level of freshness was requested and with which level it was 
ultimately fulfilled. Furthermore, for successfully executed queries that considered some kind of freshness it shall be added in the SQL response that is has been 
executed by means of a specific freshness level.
In such a way a user can not only steer its desire but is also informed whether the request could be fulfilled.

\subsection{Freshness Query Specification}

\section{Freshness Detection and extraction}

\todo{Which possibilities do we have to select a query with freshness}

\todo{How does the freshness respresentation look like for SQL.}

\todo{WITH FRESHNESS}
This can be used if a user does not require a minimum level of freshness. This can be easily supported  


\subsection{Freshness Filtering}


\subsection{Freshness Selection}


\todo{Maybe not an entire section. Rather discussed internally}
\tocless\subsection{Referential Integrity}

\section{Placement constraints}
\todoMissing{What happens for data DDL operations, to they neglect the freshness}

