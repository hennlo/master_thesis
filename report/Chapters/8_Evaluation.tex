% !TEX root = ../Thesis.tex
\chapter{Evaluation}
\label{c:evaluation}

This chapter is separated into two sections. The first section aims to validate and verify the implementation in terms of its
correct execution. It focuses on the core contributions, such as the lazy replication algorithm as well as the freshness filter capability.
Further, it ensures the correct handling of the described constraints and establishes certain test cases to identify possible failures that can occur 
during execution as well as suitable failure handling scenarios.

The second part of this chapter focuses on benchmarking the implementation based on the individual building blocks themselves.
This is followed by comparing the performance of several scenarios to identify how the system will behave in certain situations.
Further, it will compare these executions to determine how the implementation fulfills the provided requirements.
Finally, the performance of the individual functionalities is benchmarked and compared jointly to give a comprehensive overview on possible performance impacts.


\section{Goal}
The evaluation has three goals: The first goal is to verify the correctness of the solution, the second goal
is identifying the impact freshness-aware aspects have on different workloads and finally the identification of suitable workloads for freshness-related queries.

We therefore want to verify and validate the correctness as well as the completeness of the implementation based on several characteristics.
These include the correct execution of the automatic lazy replication procedure, the possibility to refresh statements on demand as well as correct freshness filtering.

Since one of the main goals was to relax the consistency to allow parallel workloads,
we want to identify possible impacts of the replication engine on the overall performance if freshness indeed increases the possibility
for parallel workload on the system. 
Finally, we want to observe if the solution allows harvesting benefits of the underlying stores to adapt to various situations
to assess the benefits of the Polystore in general.
We therefore, focus on evaluating the essential functionalities to provide a foundation for future assessments.


%%%%%%%%%%%%%%%%%%%%%%%%%%%%%%%%%%%%%%%%%%%%%%%%%%%%%%%%%%%%%%%%%%

\section{Correctness}
\label{sec:correctness}


The correctness of the introduced solution mainly focuses on two parts. For one the replication behavior, to verify if each replication is carried out correctly
and if not verify that reasonable countermeasures are available and apply them. This is crucial since we do not compare the footprint or the integrity of the data after 
a replication update. Rather we compare metadata of two replicas on a high level,
i.e. if the number of modifications and the commit timestamp after the data replication are equal on the primary and secondary node. 

The second part of the validation process focuses on the retrieval of outdated nodes. Although we always have the possibility to fallback to the primary placements as described in 
Section \ref{sec:fresh_select}, we still want to avoid excessive locking to parallelize requests to ultimately speed up the average response time.

As mentioned within the implementation chapter, several constraints will already be enforced with the execution itself (see \ref{sec:constraints}). 
We have e.g. established a dependency between the replication strategy as well as the designated placement state. When altering the strategy of an object we
automatically ensure that the state is altered as well. This does not only ensure that the placements will be addressed correctly, but also ensures that no data is lost.
For example, it enforces that a lazy replicated placement, that still has pending replications, cannot be switched to now receive updates eagerly.

Furthermore, since we have to make sure that no data is lost in general, we have established a strong dependency queue that will apply the update operation-wise in terms of the execution order.
Since we cannot avoid that stores might fail, we have ensured that failing updates will not block the replication queue or will be applied in atheorder.
Accompanied with a failure-threshold and the placement role \emph{INIFINITELY OUTDATED}, we can instruct the system to mark this placement as permanently outdated 
avoiding further updates. 
This ensures that all statements labeled as \emph{INIFINITELY OUTDATED} will always be treated correctly.

The freshness specification in general ensures that only acceptable values are considered for the respective evaluation type to propose the correct results and translate them into
a given tolerance value.
Since freshness operations violate the ACID guarantees and we will read stale data, we ensured that these can only be executed within read-only operations, 
such that they do not override data.


Additionally, to guarantee the correctness of the implementation, several unit-tests have 
been created to ensure the correct handling of data, when using freshness-awareness or replication approaches in general.
These logically include a purposely false execution of the described constraints to ensure that they will be treated correctly.



%%%%%%%%%%%%%%%%%%%%%%%%%%%%%%%%%%%%%%%%%%%%%%%%%%%%%%%%%%%%%%%%%%

\section{Benchmarks}

The second part of this chapter focuses on the performance impacts of the introduced solutions.
The following steps outline the procedure for benchmarking data freshness within Polypheny-DB.\\
Each evaluation step will subsequently address each required building block to provide data freshness.
Each building block is checked separately, and then put into perspective of the entire context.
These benchmarks progressively build on top of each other and are finally summarized with a compound benchmark.\\




\subsection{Evaluation Environment}
\label{sec:env}

For all executed tests a base set of functionalities has been used to ensure the reproducibility of the benchmarks.

If not explicitly stated otherwise, the benchmarks will be executed using two underlying stores.
For these stores, HSQLDB and PostgreSQL are used. While HSQLDB will run natively on the system, PostgreSQL is configured within a virtualized
container environment with limited resources imitating a weaker performing store.
Although not explicitly mentioned, each test will always be pre-configured to match the required replication strategy.





During the evaluation, only a predefined set of workloads will be used to obtain insights on various setups.
As suggested in Table \ref{tab:workload table} these inherently differ in terms of their focus on write- or read-operations as well the
degree of utilized freshness queries.\\


\begin{table}[h]
    \centering
    \def\arraystretch{1.5}
    \begin{tabular}{V{2} l V{2} c|c|c V{2}}
    \hlineB{3}
    \textbf{Workload}        & \textbf{Read \%} & \textbf{Write \%} & \textbf{Freshness \%}  \\ \hlineB{2}
    \textbf{Write Only}      & 0                & 100               & 0                             \\ \hline
    \textbf{Mixed Workload}  & 60               & 40         & 0                             \\ \hline
    \textbf{Read Only}       & 100              & 0                 & 0                             \\ \hline
    \textbf{Freshness Only} & 100              & 0                 & 100                           \\ \hline
    \textbf{Mixed Freshness} & 60              & 40                 & 100                           \\ \hline
    \textbf{Partial Freshness} & 100              & 0                 & 50                           \\ \hlineB{2}    
    \end{tabular}
    \caption{Available Benchmark Workloads}
    \label{tab:workload table}
\end{table}



To further assist the execution of the benchmark tests, a number of tools are 
used to standardize and ease the evaluation procedure. These tools are described in the following two sections.

\subsubsection{Chronos}

\textit{Chronos}\footnote{https://chronos-eaas.org/} is a toolkit which enables users to define, 
monitor and analyze the results of an evaluation for several database systems~\cite{vogt_chronos_2020}.
It encompasses several tests and evaluation steps, which will be accumulated in terms of the runtime 
per target. This runtime information can then be used to compare and visualize the differences between different execution environments.  



\subsubsection{OLTPBench}

For all considered functional tests \textit{OLTPBench}\footnote{https://github.com/oltpbenchmark/oltpbench} 
has been used. This benchmarking framework bundles several workload classes and 
encompasses multiple benchmarking tools to seamlessly provide benchmarking for various DBMS~\cite{oltp_2013}.\\
To get a more in-depth view of how the implementation affected the overall performance, all tests have been executed 
using \textit{YCSB}\footnote{https://github.com/brianfrankcooper/YCSB}.
\textit{YCSB} not only offers a fundamental correctness test but additionally simulates workload on a single table to obtain essential performance baselines~\cite{ycsb_2010}.
These can be used to retrieve performance metrics and obtain indications regarding correct statement execution and query handling.


%%%%%%%%%%%%%%%%%%%%%%%%%%%%%%%%%%%%%%%%%%%%%%
%%%%%%%%%%%%%%%%%%%%%%%%%%%%%%%%%%%%%%%%%%%%%%

\subsection{Results}
\label{sec:results}
In this section, we will present the results of the performance evaluation.
If not stated otherwise the generated tables within the benchmarks will contain 100000 entries and will be executed using
50 parallel client threads on YCSB together with the workloads defined in Table \ref{tab:workload table}.
During the evaluation, it will only be referred to the respective workload names.
  

           

\subsubsection{Overhead} 

\begin{figure}[t] 
    \centering 
    \begin{tikzpicture}
        \begin{axis}[
            width  = 0.85*\textwidth,
            height = 7cm,
            x=120pt,
            major x tick style = transparent,
            ybar=.2cm,
            minor y tick num=4,
            bar width=30pt,
            ymajorgrids = true,
            ylabel = {Runtime (ms)},
            xlabel = {Number of Stores},
            symbolic x coords={1,2},
            xtick = data,
            scaled y ticks = false,
            enlarge x limits=0.40,
            ymin=0,
            major tick style={thick},
            legend pos=outer north east,
            legend cell align=left,
        ]
            \addplot[style={bblue,fill =bblue,mark=none}]
                coordinates {(1, 1454546 ) (2,1712337)};

            \addplot[style={rred,fill=rred,mark=none}]
                coordinates {(1, 1569320) (2,1629462) };

            \legend{Old,New}
        \end{axis}
    \end{tikzpicture}
\caption{Overhead Comparison of Old vs. New Implementation.}
\label{fig:overhead}
\end{figure}

Polystore systems and specifically Polypheny-DB uniformly collect all incoming requests, process them and, then
route the resulting queries to the designated stores, where they will be finally executed. 
However, due to this centralized processing, additional introduced overhead within this layer will directly impact the performance of the system.
Although, Polypheny-DB aims to provide cost- and workload-aware self-adaptiveness, to provide the best possible query results,
its internal processing is executed on top of the actual execution of the underlying store.
This can have a crucial impact on the entire throughput of the system.\\
Hence we used Chronos, a benchmarking tool that allows us to define several evaluation steps per storage engine,
to measure the overhead. For this evaluation, the introduced implementation is compared against the current version of Polypheny-DB (v0.7.0).
This was executed using a fixed repeating number of operations and measured based on the total execution time that it took for each configuration
to apply all steps.


As depicted in Figure \ref{fig:overhead}, the results show that the new implementation introduced a little overhead of about 8\% for single-store operations.
However, with multiple stores the actual execution time was indeed slightly reduced by 5\%. 
Although, single store executions should not be entirely neglected
they do not form the main pursuit of Polystore systems, which are more suitably visualized by the multistore runtime.





%%%%%%%%%%%%%%%%%%%%%%%%%%%%%%%%%%%%%%%%%%%%%%%%%%%%%%%%%%%%%%%%%%

\subsubsection{Locking-Mechanism} 


\begin{figure}[t]
    \centering
    \begin{subfigure}{.5\textwidth}
      \centering
      \begin{tikzpicture}
        \begin{axis}[
            width  = 0.9*\textwidth,
            height=5cm,
            ymajorgrids = true,
            minor y tick num=3,
            ylabel = {Throughput (requests/sec)},
            xlabel = {Number of Partitions ($n$)},
            symbolic x coords={2,4,8},
            xtick = data,
            scaled y ticks = false,
            enlarge x limits=0.40,
            ymin=60,
            ymax=140,
            legend style={
                    at={(1,1.05)},
                    anchor=south east,
                    column sep=1ex
            },
            legend cell align=left,
            smooth
        ]
            \addplot[style={bblue,mark=*}]
                coordinates {(2, 82.66787244) (4,110.0694971) (8,124.4930924)};

            \addplot[style={rred,mark=x}]
                coordinates {(2, 90.32423012) (4,85.38185074) (8,69.04305019)};

            \legend{Single Store,$n$ Stores}
        \end{axis}
    \end{tikzpicture}
      \caption{Old Locking}
      \label{fig:oldlock}
    \end{subfigure}%
    \begin{subfigure}{.5\textwidth}
      \centering
      \begin{tikzpicture}
        \begin{axis}[
            width  = 0.9*\textwidth,
            height=5cm,
            ymajorgrids = true,
            minor y tick num=3,
            ylabel = {Throughput (requests/sec)},
            xlabel = {Number of Partitions ($n$)},
            symbolic x coords={2,4,8},
            xtick = data,
            scaled y ticks = false,
            enlarge x limits=0.40,
            ymin=60,
            ymax=140,
            major tick style={thick},
            legend style={
                    at={(1,1.05)},
                    anchor=south east,
                    column sep=1ex
            },
            legend cell align=left,
            smooth
        ]
            \addplot[style={bblue,mark=square*}]
                coordinates {(2, 83.56813262) (4,118.5016919) (8,136.0429577)};

            \addplot[style={rred,mark=diamond*}]
                coordinates {(2, 75.87686445) (4,90.53951027) (8,68.78432784)};

            \legend{Single Store,$n$ Stores}
        \end{axis}
    \end{tikzpicture}
      \caption{New Locking}
      \label{fig:newlock}
    \end{subfigure}

    \vspace{10pt}%

    \begin{subfigure}{.7\textwidth}
        \centering
        \begin{tikzpicture}
            \begin{axis}[
                width  = 0.7*\textwidth,
                height=5cm,
                ymajorgrids = true,
                ylabel = {Throughput (requests/sec)},
                xlabel = {Number of Partitions ($n$)},
                symbolic x coords={2,4,8},
                xtick = data,
                scaled y ticks = false,
                enlarge x limits=0.40,
                minor y tick num=3,
                ymin=60,
                ymax=140,
                major tick style={very thick},
                legend pos=outer north east,
                legend cell align=left,
                smooth
            ]
                \addplot[style={bblue,mark=*}]
                    coordinates {(2, 82.66787244) (4,110.0694971) (8,124.4930924)};
    
                \addplot[style={rred,mark=square*}]
                    coordinates {(2, 83.56813262) (4,118.5016919) (8,136.0429577)};
    
                \legend{Old,New}
            \end{axis}
        \end{tikzpicture}
        \caption{Old vs. New Locking on a Single Store}
        \label{fig:oldandnewlock}
      \end{subfigure}
    \caption{Impact of Locking Mechanism on the Overall Throughput per Second.}
    \label{fig:lock_comp}
\end{figure}


As described in Section \ref{sec:strategy}, one of the prerequisites to establish multiple refresh strategies and hence lazy replication,  
was the refactoring of the locking mechanism. Although, not completely reworked, the locking module of Polyphenys SS2PL 
poses as a core component of the system. It therefore, impacts correct serializability treatment and is an inherent driver of 
the allowed concurrency which directly influences the overall performance of the system.\\
For the evaluation again the current state of Polypheny-DB is compared against this implementation.
Since the locking module was changed from a table-wise locking to a partition-wise locking we will validate the impact on the basis of 
a single table using YCSB. 
The evaluation was executed with gradually increasing numbers of partitions, which are placed on one store or distributed across $n$-Stores 
for $n$ partitions to observe any changes in the locking and therefore the throughput.
To get a general overview of the impact, the benchmark was executed using a mixed workload on HSQLDB only.
It is evaluated on the number of operations that can be applied to the system per second.




As visualized in Figures \ref{fig:oldlock} and \ref{fig:newlock}, for both cases the overall situation is quite similiar.
While the distribution of the partitions across several stores gets gradually worse, the single store performance actually improves, the more partitions are added to the table.
This behavior is essentially caused by Polyphenys need to join and union several stores together, when querying multiple partitions across several stores.
Since more stores need to be connected and considered, it is a rather costly approach and as stated before gets increasingly harder the more stores are involved.

Because the single store variations prove to be more reliable, they are summarized in Figure \ref{fig:oldandnewlock}.
We can observe that the new locking mechanism, indeed proves to be performing around 9\% better in terms of the possible throughput, compared to the old implementation.
Furthermore, it shows that again with a growing number of partitions, the gap between the old and new locking extends even more, 
validating the benefits of the new locking mechanism, if used within strongly partitioned configurations.




%%%%%%%%%%%%%%%%%%%%%%%%%%%%%%%%%%%%%%%%%%%%%%%%%%%%%%%%%%%%%%%%%%


\subsubsection{Baseline Identification} 

%%%%%%%%%%%%%%%%%%%%%%%%%%%%%%%%%%%%%%%%%
% Single Store performance of HSQL and PSQL
%%%%%%%%%%%%%%%%%%%%%%%%%%%%%%%%%%%%%%%%%



\begin{figure}[t] 
    \centering 
    \begin{tikzpicture}
        \begin{axis}[
            width  = 0.85*\textwidth,
            height = 6cm,
            x=80pt,
            major x tick style = transparent,
            ybar=.2cm,
            bar width=30pt,
            ymajorgrids = true,
            ylabel = {Avg. Response Time (ms)},
            xlabel = {Stores},
            symbolic x coords={HSQLDB,PostgreSQL},
            xtick = data,
            scaled y ticks = false,
            enlarge x limits=0.50,
            minor y tick num=4,
            ymin=0,
            ymax=3000,
            major tick style={thick},
            legend pos=outer north east,
            legend cell align=left,
        ]
            \addplot[style={bblue,fill=bblue,mark=none}]
                coordinates {(HSQLDB, 2210 ) (PostgreSQL,2700)};

        \end{axis}
    \end{tikzpicture}
    \caption{Distinct Write Time Comparison of HSQLDB and PostgreSQL.}
    \label{fig:singlepsqlhsql}
\end{figure}

The Lazy Replication algorithm is not only fundamental to generate multiple versions to be used within freshness-awareness,
but it is also a core functionality of how data is propagated throughout the system. 
Hence, along with the newly introduced replication strategies, and \emph{Change Data Collection} it will have a major impact on the overall performance of the system.
Since the replication solely focuses on replicating captured changes, the next benchmarks will be consequently executed using Write Only operations without any reads.


As stated in the evaluation environment Section \ref{sec:env}, these benchmarks will be mainly executed with an embedded version of HSQLDB and PostgreSQL running within a virtualized 
container environment. To have a general baseline for comparison Figure \ref{fig:singlepsqlhsql} presents a single store execution, comparing these two stores against each other.
As motivated in the beginning, it is crucial for a Polystore system to utilize the key benefits of each store to provide the best results. For our scenario, it is therefore important to determine which 
store configuration is more suitable to be used as an eagerly replicated primary placement, due to its lower latency and better response time.\\
This illustrated comparison clearly shows that due to its limited resources the PostgreSQL store
performs on average 22\% slower and cannot directly compete with this HSQLDB store.
This provides us with the intuitive decision to use HSQLDB for the primary transactions.




%%%%%%%%%%%%%%%%%%%%%%%%%%%%%%%%%%%%%%%%%%%%%%%%%%%%%%%%%%%%%%%%%%




\subsubsection{Lazy Replication} 

%%%%%%%%%%%%%%%%%%%%%%%%%%%%%%%%%%%%%%%%%
% Terminal 1 vs. Terminal 50
%%%%%%%%%%%%%%%%%%%%%%%%%%%%%%%%%%%%%%%%%

\begin{figure}[t]
    \centering
    \begin{subfigure}{.5\textwidth}
      \centering
      \begin{tikzpicture}
        \begin{axis}[
            width  = 0.7*\textwidth,
            x=80pt,
            major x tick style = transparent,
            ybar=.2cm,
            bar width=20pt,
            ymajorgrids = true,
            minor y tick num=4,
            ylabel = {Throughput (requests/sec)},
            xlabel = {Replication Strategy},
            symbolic x coords={Eager,Lazy},
            xtick = data,
            scaled y ticks = false,
            enlarge x limits=0.40,
            ymin=0,
            ymax=30,
            major tick style={thick},
            legend style={
                    at={(1,1.05)},
                    anchor=south east,
                    column sep=1ex
            },
            legend cell align=left,
        ]
            \addplot[style={bblue,fill =bblue,mark=none}]
                coordinates {(Eager, 26.15944582) (Lazy,19.0566423)};

            \addplot[style={rred,fill=rred,mark=none}]
                coordinates {(Eager, 16.9502111) (Lazy,11.78997423) };

            \legend{HSQLDB,PostgreSQL}
        \end{axis}
    \end{tikzpicture}
      \caption{One process}
      \label{fig:terminal1}
    \end{subfigure}%
    \begin{subfigure}{.5\textwidth}
      \centering
      \begin{tikzpicture}
        \begin{axis}[
            width  = 0.7*\textwidth,
            x=80pt,
            major x tick style = transparent,
            ybar=.2cm,
            bar width=20pt,
            ymajorgrids = true,
            minor y tick num=4,
            ylabel = {Throughput (requests/sec)},
            xlabel = {Replication Strategy},
            symbolic x coords={Eager,Lazy},
            xtick = data,
            scaled y ticks = false,
            enlarge x limits=0.40,
            ymin=0,
            ymax=30,
            major tick style={thick},
            legend style={
                    at={(1,1.05)},
                    anchor=south east,
                    column sep=1ex
            },
            legend cell align=left,
        ]
            \addplot[style={bblue,fill =bblue,mark=none}]
                coordinates {(Eager, 19.08643067 ) (Lazy,14.62659111)};

            \addplot[style={rred,fill=rred,mark=none}]
                coordinates {(Eager, 15.52999761) (Lazy,11.86666016) };
                

                \legend{HSQLDB,PostgreSQL}
        \end{axis}
    \end{tikzpicture}
      \caption{50 processes}
      \label{fig:terminal50}
    \end{subfigure}
    \caption{Throughput Impact of Concurrency in a Replicated Setup.}
    \label{fig:terminal}
\end{figure}

As previously stated, the replication strategies will impact the processing capability of the system immensely.
A placement with a configured lazy replication strategy automatically enables the system, to start tracking changes for this entity, impacting the duration of a query.
Therefore, we want to compare how each store handles the replication. Consequently, we want to benchmark and compare two equal placements that are eagerly replicated
against the same two stores but one configured as \emph{lazy}.

To extend the baseline discovered before, we again want to demonstrate the behavior a purely sequential environment with only one client has, 
against a parallel environment with 50 clients.
Figure \ref{fig:terminal} shows the evaluation across two stores, providing the possible throughput per second. This is given as the number of modifications that can be 
applied to the system per second. As before HSQLDB achieves better results than PostgreSQL. Furthermore, disregarding the underlying store,
the eagerly replicated configuration performs much better in all tests, which is not directly apparent when only considering Figure \ref{fig:terminal50}.
However, considering that the collection of changes within a lazy setup, indeed imposes additional costs on the processing time, such deviations are expected.


Admittingly an entity that is composed of only similar or equal stores, will not be beneficial for a Polystore system, to allow different workloads.
Therefore the following benchmarks will concentrate on a mixed setup with interleaved stores. These tests will be executed with 50 parallel clients to reproduce a 
conventional environment. 

%%%%%%%%%%%%%%%%%%%%%%%%%%%%%%%%%%%%%%%%%%%%%%%%%%%%%%%%%%%%%%%%%%

%%%%%%%%%%%%%%%%%%%%%%%%%%%%%%%%%%%%%%%%%
% HSQL and PSQL vs. Eager
%%%%%%%%%%%%%%%%%%%%%%%%%%%%%%%%%%%%%%%%%
\subsubsection{Interleaved Configurations}
Now focussing on a mixed setup of two stores containing PostgreSQL as well as HSQLDB for one entity.
Nativiely for a Polystore environment we want to identify which setup of underlying stores will produce better results hence is suitable for which situation.
Consequently, we want to observe how the order of the stores impacts the response time per benchmark.\\
Again to have a foundation to compare our changes to, we will compare the configurations if both stores are defined as eager and Further
respectiviely define each store as lazy as well.\\
Ultimately Figure \ref{fig:psqlhsqlresponse} shows that regarding the lazy approaches, we again see the common behavior that HSQLDB performs a little better as an eagerly replicated
store compared to PostgreSQL.


\begin{figure}[t] 
    \begin{tikzpicture}
        \begin{axis}[
            width  = 0.9*\textwidth,
            height=6cm,
            ymajorgrids = true,
            ylabel = {Avg. Response Time (ms)},
            xlabel = {Time (sec)},
            scaled y ticks = false,
            minor y tick num=4,
            minor x tick num=9,
            xmin=0,
            xmax=300,
            ymin=2000,
            ymax=5000,
            grid,
            grid style={dotted},
            major tick style={very thick},
            legend style={
                        at={(1,1.05)},
                        anchor=south east,
                        column sep=1ex
                },
            legend cell align=left,
        ]
            \addplot[style={bblue,mark=*}, mark options={scale=0.5}] 
                table [x=time, y=hsql]{Plot/psqlhsqlresponse.dat};
    
            \addplot[style={rred,mark=x}, mark options={scale=0.5}] %[thick]
                table [x=time, y=psql]{Plot/psqlhsqlresponse.dat};
    
            \addplot[style={ppurple,mark=diamond*}, mark options={scale=0.5}] %[thin]
                table [x=time, y=both]{Plot/psqlhsqlresponse.dat};     
    
            \legend{   
                HSQLDB (Eager) - PostgreSQL (Lazy),   
                PostgreSQL (Eager) - HSQLDB (Lazy),           
                PostgreSQL (Eager) - HSQLDB (Eager)
                }
        \end{axis}
    \end{tikzpicture}
    \caption{Response Time Comparsion of Various Store Configurations -- Write Only.}
    \label{fig:psqlhsqlresponse}
    \end{figure}



    \begin{figure}[t] 
        \centering 
        \begin{tikzpicture}
            \begin{axis}[
                width  = 0.7*\textwidth,
                x=80pt,
                major x tick style = transparent,
                ybar=.2cm,
                bar width=20pt,
                ymajorgrids = true,
                ylabel = {Avg. Response Time (ms)},
                xlabel = {Store Compositions},
                symbolic x coords={1},
                xtick = data,
                xticklabels={},
                scaled y ticks = false,
                enlarge x limits=0.40,
                ymin=0,
                ymax=4500,
                minor y tick num=4,
                major tick style={thick},
                legend cell align=left,
                legend pos=outer north east
            ]
                \addplot[style={bblue,fill =bblue,mark=none}]
                    coordinates {(1, 2598.875133)};    
                
                \addplot[style={rred,fill=rred,mark=none}]
                    coordinates {(1, 3194.893767)};    
    
                \addplot[style={ggreen,fill =ggreen,mark=none}]
                    coordinates {(1, 2948.031083)};
    
                \addplot[style={ppurple,fill=ppurple,mark=none}]
                    coordinates {(1, 4018.661833)};
    
                \addplot[style={orange,fill =orange,mark=none}]
                    coordinates {(1, 3475.416683)};
    
                \legend{ 
                    HSQLDB x2 (Eager),
                    PostgreSQL x2 (Eager),
                    PostgreSQL (Eager) - HSQLDB (Eager),
                    PostgreSQL (Eager) - HSQLDB (Lazy), 
                    HSQLDB (Eager) - PostgreSQL (Lazy),
                    }
            \end{axis}
        \end{tikzpicture}
        \caption{Response Time Comparsion of Various Store Compositions -- Write Only.}
        \label{fig:overall_comp}
    \end{figure}


Additionally, the Figure in \ref{fig:overall_comp} puts the average latency in perspective to the execution times described before.
As one can observe the eager replications again peform the best while the PostgreSQL variation posing as eager, performs the worst.
This not only allows us to compare the different setups but again aids us to choose suitable store combinations to be used for our designated tasks.\\






%%%%%%%%%%%%%%%%%%%%%%%%%%%%%%%%%%%%%%%%%
% Store Scale 2 vs. 4 HSQL and PSQL
%%%%%%%%%%%%%%%%%%%%%%%%%%%%%%%%%%%%%%%%%

However, the presented possibilities so far only considered the execution on two stores. Therefore, Figure \ref{fig:24storecomp} aims
to compare the execution on two and four stores to identify any deviations the lazy replication algorithm has. 
Again this is done in an interleaved fashion, switching the role of lazy and eager strategy between the participating stores.
During these tests only one is eagerly replicated, the remaining stores are all configured as lazy placements. Eagerly and lazily replicated stores
in this scenario are defined to be different store types.\\
The graphs indicate that although they differ in terms of average response times, the gap between both configurations is comparably equal.
In both cases, the execution with the eagerly replicated HSQLDB is roughly 500ms faster in terms of the average response time.\\

As summarized and visualized more densely in Figure \ref{fig:24storecomp_avg},
its rather counter-intuitive that the configurations in Figure \ref{fig:24storecomp} (a) and (b) deviate at all. 
In general, they both only contain one primary placement that is even targeted for the primary transaction.
All other stores will be updated asynchronously and are therefore not directly involved.
However, as described before, the primary transaction is also responsible for capturing, as well as queuing the changes to be replicated asynchronously.
While the procedure is always executed equally, the second approach with four stores has more replication targets that require the change. 
Since the generation of replication objects as well as the queueing are all still done during the commit of the primary transaction, the observed
deviations are reasonable.\\

\begin{figure}[t]
    \centering

    \begin{tikzpicture}
        \begin{customlegend}[
            legend columns=2,
            legend style={align=left,column sep=2ex},
            legend entries={HSQLDB (Eager) - PostgreSQL (Lazy),
                            PostgreSQL (Eager) - HSQLDB (Lazy)
                            }]
            \addlegendimage{mark=none,solid,bblue, line legend}
            \addlegendimage{mark=none,rred, solid}   
            \end{customlegend}
        \end{tikzpicture}

        \vspace{10pt}%

    \begin{subfigure}{.5\textwidth}
      \centering
      \begin{tikzpicture}
        \begin{axis}[
            width  = 0.95*\textwidth,
            height=5cm,
            ymajorgrids = true,
            ylabel = {Avg. Response Time (ms)},
            xlabel = {Time (sec)},
            scaled y ticks = false,
            minor y tick num=4,
            minor x tick num=9,
            xmin=0,
            xmax=300,
            ymin=2000,
            ymax=7000,
            grid,
            grid style={dotted},
            major tick style={very thick},
            legend style={
                        at={(1,1.05)},
                        anchor=south east,
                        column sep=1ex,
                        text width=3.5cm
                },
            legend cell align=left,
        ]
            
    
            \addplot[style={bblue,mark=none}, mark options={scale=0.5}] 
            table [x=time, y=hsql_1]{Plot/24storecomp.dat};

            \addplot[style={rred,mark=none}, mark options={scale=0.5}] %[thick]
                table [x=time, y=psql_1]{Plot/24storecomp.dat};      
                %\addlegendentry[minimum height=0.8cm]{HSQLDB (Eager) -\\ PostgreSQL (Lazy)}
                %\addlegendentry[minimum height=1.2cm]{PostgreSQL (Eager) -\\ HSQLDB (Lazy)}
        \end{axis}
    \end{tikzpicture}
      \caption{2 Stores}
      \label{fig:2store}
    \end{subfigure}%
    \begin{subfigure}{.5\textwidth}
      \centering
      \begin{tikzpicture}
        \begin{axis}[
            width  = 0.95*\textwidth,
            height=5cm,
            ymajorgrids = true,
            ylabel = {Avg. Response Time (ms)},
            xlabel = {Time (sec)},
            scaled y ticks = false,
            minor y tick num=4,
            minor x tick num=9,
            xmin=0,
            xmax=300,
            ymin=2000,
            ymax=7000,
            grid,
            grid style={dotted},
            major tick style={very thick},
            legend style={
                        at={(1,1.05)},
                        anchor=south east,
                        column sep=1ex,
                        text width=3.5cm
                },
            legend cell align=left,
        ]
            
    
            \addplot[style={bblue,mark=none}, mark options={scale=0.5}] 
            table [x=time, y=hsql_3]{Plot/24storecomp.dat};

            \addplot[style={rred,mark=none}, mark options={scale=0.5}] %[thick]
                table [x=time, y=psql_3]{Plot/24storecomp.dat};      
    
            %\addlegendentry[minimum height=0.8cm]{HSQLDB (Eager) -\\ PostgreSQL (Lazy) x3}
            %\addlegendentry[minimum height=1.2cm]{PostgreSQL (Eager) -\\ HSQLDB (Lazy) x3}

        \end{axis}
    \end{tikzpicture}
      \caption{4 Stores (Lazy x3)}
      \label{fig:4store}
    \end{subfigure}
    \caption{Response Time Comparison Among Different Store Sizes with Interleaved Roles -- Write Only.}
    \label{fig:24storecomp}
\end{figure}




\begin{figure}[t] 
    \centering 
    \begin{tikzpicture}
        \begin{axis}[
            width  = 0.8*\textwidth,
            height = 6cm,
            x=120pt,
            major x tick style = transparent,
            ybar=.2cm,
            bar width=30pt,
            ymajorgrids = true,
            ylabel = {Avg. Response Time (ms)},
            xlabel = {Number of Stores},
            symbolic x coords={2,4},
            xtick = data,
            scaled y ticks = false,
            enlarge x limits=0.40,
            ymin=0,
            ymax=6000,
            minor y tick num=4,
            major tick style={thick},
            legend style={
                    at={(1,1.05)},
                    anchor=south east,
                    column sep=1ex
            },
            legend cell align=left,
        ]
            \addplot[style={bblue,fill =bblue,mark=none}]
                coordinates {(2, 3653.642433) (4,5346.252567 )};

            \addplot[style={rred,fill=rred,mark=none}]
                coordinates {(2, 4018.661833) (4,5771.588467) };

            \legend{HSQLDB (Eager) - PostgreSQL (Lazy),
                PostgreSQL (Eager) - HSQLDB (Lazy) 
            }
        \end{axis}
    \end{tikzpicture}
    \caption{Avg. Response Time Comparison of Different Store Sizes (4 Stores $\rightarrow$ 1x Eager and 3x Lazy ) -- Write Only.}
    \label{fig:24storecomp_avg}
\end{figure}





%%%%%%%%%%%%%%%%%%%%%%%%%%%%%%%%%%%%%%%%%
% Store Scale 2-8
%%%%%%%%%%%%%%%%%%%%%%%%%%%%%%%%%%%%%%%%%
Based on this observation we specifically wanted to compare how growth in stores will impact this deviation and how it compares against its eager counterpart. 
To have a more uniform result this test will be executed using only HSQLDB stores, to have a stable foundation for comparison
without a second store that could interfere with the final result.\\
In this evaluation Figure \ref{fig:stores_comp} illustrates, the average response time of two to eight stores, where one store is eagerly replicated and the rest
is configured as lazy.


\begin{figure}[t] 
    \centering 
    \centering
        \begin{tikzpicture}
            \begin{axis}[
                width  = 0.6*\textwidth,
                height = 6cm,
                ymajorgrids = true,
                ylabel = {Avg. Response Time (ms)},
                xlabel = {Number of Stores},
                symbolic x coords={2,4,8},
                xtick = data,
                scaled y ticks = false,
                enlarge x limits=0.40,
                minor y tick num=3,
                ymin=0,
                ymax=6000,
                major tick style={very thick},
                legend pos=outer north east,
                legend cell align=left,
                smooth
            ]
                \addplot[style={bblue,mark=*}]
                    coordinates {(2, 2598.875133) (4,3441.368233) (8,4911.982467)};
    
                \addplot[style={rred,mark=x}]
                    coordinates {(2, 3492.3136) (4,3943.186767) (8,5532.865083)};
    
                \legend{Eager,Lazy}
            \end{axis}
        \end{tikzpicture}
    \caption{Replication Strategy Comparison with Increasing Stores -- Write Only.}
    \label{fig:stores_comp}
\end{figure}

This indicates that although the repsonse time increases, it does so in a stable manner. As compared with its eager approach we can see that it evolves roughly towards the same direction plus the
previously observed offset, generally providing good scalability.

\subsubsection{Queue Replication}

%%%%%%%%%%%%%%%%%%%%%%%%%%%%%%%%%%%%%%%%%
% Single operation comparison
%%%%%%%%%%%%%%%%%%%%%%%%%%%%%%%%%%%%%%%%%

As presented during the implementation and now elaborated and suggested multiple times during evaluation, 
the capture of modifications to be propagated to the secondary placements, introduces some overhead. Although some overhead
is reasonable due to the additional processing steps, the response times differ 
especially when compared to regular eagerly replicated scenarios. This directly impacts the overall performance,
resulting in generally higher response times for lazy replication scenarios.
Therefore we want to identify for a single write-operation what actually influences these runtime deviations.


\begin{figure}[t] 
    \centering 
    \begin{tikzpicture}
        \begin{axis}[
            width  = 0.8*\textwidth,
            height=6cm,
            ybar stacked,
            x=80pt,
            major x tick style = transparent,
            bar width=30pt,
            ymajorgrids = true,
            ylabel = {Execution Time (ms)},
            xlabel = {Replication Statements},
            symbolic x coords={Eager, Lazy},
            xtick = data,
            scaled y ticks = false,
            enlarge x limits=0.40,
            ymin=0,
            ymax=70,
            minor y tick num=4,
            major tick style={thick},
            legend pos=outer north east,
            legend cell align=left,
        ]
            \addplot+[ybar][style={bblue,fill =bblue,mark=none}]
                coordinates {(Eager, 21) (Lazy, 18)};

            \addplot+[ybar][style={rred,fill=rred,mark=none}]
                coordinates {(Eager, 0) (Lazy, 16)};

            \addplot+[ybar][style={ppurple,fill =ppurple,mark=none}]
                coordinates {(Eager, 0) (Lazy, 28)};

            \legend{ 
                Base Execution,
                Capture Queue,
                Replication,
                }
        \end{axis}
    \end{tikzpicture}
    \caption{Execution Time Comparision of a Decompossed Distinct Write-Operation with and without Active Data Capture.}
    \label{fig:write_decomposition}
\end{figure}



The comparison in Figure \ref{fig:write_decomposition} analyzes the execution time of two independent write operations, respectively executed on two equal stores.
One configuration is a simple eager replicated entity on two stores without any replication, the other one is an operation that needs to capture 
the modification within the global replication queue.\\
Since the lazy approach only needs to involve one store for processing, the baseline of the lazy approach is indeed slightly faster than its eager counterpart.
However, considering that at commit time this approach also needs to extract the captured change, convert it into a replication object and ultimately queue 
it for the remaining store, the actual execution deviates quite heavily. With the queue time being almost equal to its base execution time.\\
However, since the capture part is only executed once at the commit time of a transaction, it drastically impacts the performance 
of one single operation.
As we have seen before, this introduced capture gap remains quite stable even with additional targets or capture objects to transform.
Therefore the queueing process negatively impacts the efficiency of transactions containing only one or few operations.
In contrast, it gets neglected further the more operations are executed within one transaction.\\
Additionally, although not directly considered to be part of the execution time, is the convergence window. As mentioned before, as soon
as a replication worker will have free resources it will replicate the pending object from the queue onto the secondary store.
As we can see the replication duration takes a little bit longer than the actual execution. This is caused by the introduced
validation constraints for each worker to assess the queue and reconstruct each operation individually. 
However, since the replication is done operation-wise the replication time is considered to be fixed per operation.
Therefore, the entire bar reflects the time it takes for one operation to be executed and replicated across all participating stores,
considering the aforementioned constraints of the queue time.
Since the eagerly replicated approach is executed within one single transaction targeting both underlying stores, the entity is 
considered to be immediately consistent and will not need to converge. 






%%%%%%%%%%%%%%%%%%%%%%%%%%%%%%%%%%%%%%%%%%%%%%%%%%%%%%%%%%%%%%%%%%


\subsubsection{Replica Convergence} 



\begin{figure}[t] 
    \centering 
    \begin{tikzpicture}
        \begin{axis}[
            width  = 0.9*\textwidth,
            height = 7cm,
            x=120pt,
            major x tick style = transparent,
            ybar=.2cm,
            bar width=30pt,
            ymajorgrids = true,
            ylabel = {Execution Time (sec)},
            xlabel = {Number of Stores},
            symbolic x coords={2,4},
            xtick = data,
            scaled y ticks = false,
            enlarge x limits=0.40,
            ymin=0,
            ymax=45,
            minor y tick num=4,
            major tick style={thick},
            legend style={
                    at={(1,1.05)},
                    anchor=south east,
                    column sep=1ex
            },
            legend cell align=left,
        ]
            \addplot[style={bblue,fill =bblue,mark=none}]
                coordinates {(2, 33.4120) (4,39.18 )};

            \addplot[style={rred,fill=rred,mark=none}]
                coordinates {(2, 30.322) (4,36.9) };

            \legend{HSQLDB (Eager) - PostgreSQL (Lazy),
                PostgreSQL (Eager) - HSQLDB (Lazy) 
            }
        \end{axis}
    \end{tikzpicture}
    \caption{Convergence Time Comparison on Different Store Constellations and Sizes\\(4 Stores $\rightarrow$ 1x Eager and 3x Lazy ) -- Mixed Workload.}
    \label{fig:store_comparision}
\end{figure}

As we have seen before the versions can deviate quite heavily during a mixed workload. 
Furthermore, we have already identified that configuring the faster performing store to be replicated eagerly, prooves that this will positively impact 
the availability due to its lower response time of the initial transaction.
However as important as the throughput of the initial transaction might be, choosing the faster node to be eagerly replicated intuitively leaves the slower node
to apply the updates asynchronously. 
However, because it highly depends on the requirements of which configuration to choose, we also want to establish a setup that focuses on quickly reaching
a consistent state without using an eager-only replication setup.



Therefore, we want to observe the actual replication time until the two stores reach an equilibrium in terms of their received updates.\\
The plot in Figure \ref{fig:store_comparision} visualizes the comparison based on a different number of stores.

Due to the fact that HSQLDB was before identified as the store with the better throughput, 
it also generally manages to converge faster if configured to be the secondary store. 
Since HSQLDB in our scenario can replicate the operations faster than the primary transaction can queue new events,
it does not only converge faster but has a lower execution time in general.\\
Therefore we have established that with our introduced algorithm, indeed
the better performing store is also more suitable when we want to reach consistency faster, hence providing a smaller convergence time.\\




Although the convergence time is not essentially impacted by the executing store, but also by the number of designated stores that shall asynchronously 
receive the operation. 
Further, we want to identify how the queue behaves if the secondary system is generally slower and therefore not able to replicate 
the pending changes as fast as new ones are added to the queue.



  \begin{figure}[t]
    \centering

    \begin{tikzpicture}
        \begin{customlegend}[
            legend columns=2,
            legend style={align=left,column sep=2ex},
            legend entries={Response Time (ms),
                            Replication Queue
                            }]
            \addlegendimage{mark=none,solid,bblue, line legend}
            \addlegendimage{mark=none,rred, solid}   
            \end{customlegend}
        \end{tikzpicture}

    \begin{subfigure}{.5\textwidth}
      \centering
      
        \begin{tikzpicture}
        \begin{axis}[
            scale only axis,
            width  = 0.6*\textwidth,
            axis y line*=left,% the ’*’ avoids arrow heads
            height=3cm,
            ylabel = {Avg. Response Time (ms)},
            xlabel = {Time (sec)},
            scaled y ticks = false,
            minor y tick num=4,
            minor x tick num=4,
            xmin=0,
            xmax=340,
            ymin=2000,
            ymax=6000,
            ylabel shift=-4pt,
            restrict x to domain=0:305,
            grid,
            grid style={dotted},
            major tick style={very thick},
            legend style={
                            at={(1,1.05)},
                            anchor=south east,
                            column sep=1ex
                    },
            legend cell align=left,
        ]
        \addplot[style={bblue,mark=none}, mark options={scale=0.5}] 
                    table [x=time, y=psql_1]{Plot/converge24.dat};
            
        \end{axis}
        \begin{axis}[
            scale only axis,
            width  = 0.6*\textwidth,
            height=3cm,
            xmin=0,
            xmax=340,
            ylabel shift = -4pt,
            ymin=0,
            ymax=800,
            axis y line*=right,
            axis x line=none,
            ylabel=Queue Size
          ]
          \addplot[style={rred, fill=rred, fill opacity=0.4, mark=none}, mark options={scale=0.5}] %[thick]
                    table [x=time, y=queue_2]{Plot/converge24.dat};
        \end{axis}
      \end{tikzpicture} 
      \caption{2 Stores}
      \label{fig:converge_2}
    \end{subfigure}\hfill% 
    \begin{subfigure}{.5\textwidth}
      \centering
      \begin{tikzpicture}
        \begin{axis}[
            scale only axis,
            width  = 0.6*\textwidth,
            height=3cm,
            axis y line*=left,% the ’*’ avoids arrow heads
            ylabel = {Avg. Response Time (ms)},
            xlabel = {Time (sec)},
            scaled y ticks = false,
            minor y tick num=4,
            minor x tick num=4,
            xmin=0,
            xmax=405,
            ylabel shift=-4pt,
            ymin=2000,
            ymax=6000,
            restrict x to domain=0:310,
            grid,
            grid style={dotted},
            major tick style={very thick},
        ]
        \addplot[style={bblue,mark=none}, mark options={scale=0.5}] 
                    table [x=time, y=psql_3]{Plot/converge24.dat};
                
        \end{axis}
        \begin{axis}[
            scale only axis,
            width  = 0.6*\textwidth,
            height=3cm,
            xmin=0,
            xmax=405,
            ymin=0,
            ymax=4000,
            ylabel shift = -4pt,
            axis y line*=right,
            axis x line=none,
            ylabel=Queue Size
          ]
          \addplot[style={rred, fill=rred, fill opacity=0.4, mark=none}, mark options={scale=0.5}] %[thick]
                    table [x=time, y=queue_4]{Plot/converge24.dat};

        \end{axis}
      \end{tikzpicture}
      \caption{4 Stores}
      \label{fig:converge_4}
    \end{subfigure}
    \caption{Execution Time along the Replication Queue Convergence -- Mixed Workload.}
    \label{fig:converge_24}
\end{figure}




%%%% %% %%



\begin{figure}[t] 
    \centering 
    \begin{tikzpicture}
        \begin{axis}[
            width  = 0.8*\textwidth,
            height=5cm,
            ymajorgrids = true,
            ylabel = {Queue Size},
            xlabel = {Time (sec)},
            scaled y ticks = false,
            minor y tick num=4,
            minor x tick num=4,
            xmin=0,
            xmax=405,
            ymin=0,
            ymax=4000,
            grid,
            grid style={dotted},
            major tick style={very thick},
            legend pos=outer north east,
            legend cell align=left, 
        ]
            \addplot[style={bblue, fill=bblue, fill opacity=0.6, mark=none}, mark options={scale=0.5}] 
                table [x=time, y=queue_2]{Plot/converge24.dat};
    
            \addplot[style={rred, fill=rred, fill opacity=0.4, mark=none}, mark options={scale=0.5}] %[thick]
                table [x=time, y=queue_4]{Plot/converge24.dat};
      
    
            \legend{   
                2 Stores,   
                4 Stores
                }
        \end{axis}
    \end{tikzpicture}
    \caption{Replication Queue Convergence Over Time -- Mixed Workload.}
    \label{fig:converge24}
\end{figure}



For this test, we are using a Mixed Workload to allow permanent access to the primary node (HSQLDB), while the lazy replicated nodes (PostgreSQL) can solely concentrate on applying the queued changes.
Figure \ref{fig:converge_24} visualizes the expansion- as well as the shrinking-phase of the replication queue
and puts it into perspective along with the actual test execution. During these tests, one worker continuously processed the queue and replicated 
the changes operation-wise. As illustrated in both cases the worker is not able to apply the changes faster than they are coming in.
The breakeven point is reached as soon as the test execution has finished and the load on the system stops. For both configurations, the algorithm can now quickly propagate all pending
changes.



As mentioned before the accumulation of replication events is essentially caused by the different performance capabilities of the underlying stores. 
While HSQLDB as a lazy replicated store would process and apply the events faster, 
leaving the queue mostly empty, PostgreSQL is not able to match this performance by applying the replication events slower than new ones
are generated. This then causes the queue to grow until there is no more load on the system.
However, without a defined test enviornment in regular situations with continuous load, this could cause severe problems,
since the queue will grow without bound.




Because captured change operations are also transformed $1-n$ for $n$ lazy replicated stores, the size of the queues is further influenced by the number of replicas that
require the change. 
As summarized in Figure \ref{fig:converge24} this directly influences the earlier described convergence time of our system and negatively impacts the response time.
While the configuration with two stores only needs an additional 30 seconds (~10\% of total execution time) to converge, a setup with four stores
already needs 105 seconds (~35\%) to converge the remaining stores.
Which makes the selected storage constellation crucial.









%%%%%%%%%%%%%%%%%%%%%%%%%%%%%%%%%%%%%%%%%%%%%%%%%%%%%%%%%%%%%%%%%%
\subsubsection{Freshness Evaluation Type Filter} 

The benchmarks so far merely focussed on the replication as well as the write-operations.
Therefore the next evaluations will concentrate on testing the introduced freshness capabilities.

Despite the actual usage of freshness operations in general, also the chosen evaluation type will influence the performance.
Although, that the filter comparison itself is always executed equally, the filter generation can differ between the available types.

\begin{figure}[t] 
    \centering 
    \begin{tikzpicture}
        \begin{axis}[
            width  = 0.8*\textwidth,
            height=6cm,
            x=80pt,
            major x tick style = transparent,
            ybar=.2cm,
            bar width=25pt,
            ymajorgrids = true,
            ylabel = {Avg. Response Time (ms)},
            xlabel = {Evaluation Types},
            symbolic x coords={1},
            xtick = data,
            xticklabels={},
            scaled y ticks = false,
            enlarge x limits=0.40,
            ymin=0,
            ymax=3500,
            minor y tick num=4,
            major tick style={thick},
            legend pos=outer north east,
            legend cell align=left,
        ]
            \addplot[style={bblue,fill =bblue,mark=none}]
                coordinates {(1, 3000)};

            \addplot[style={rred,fill=rred,mark=none}]
                coordinates {(1, 3190)};

            \addplot[style={ggreen,fill =ggreen,mark=none}]
                coordinates {(1, 3350)};

            \addplot[style={ppurple,fill=ppurple,mark=none}]
                coordinates {(1, 3150)};

            \legend{ 
                Timestamp,
                Absolute Delay,
                Relative Delay,
                Freshness Index
                }
        \end{axis}
    \end{tikzpicture}
    \caption{Avg. Response Time Comparison of Available Evaluation Types -- Freshness Only}
    \label{fig:eval_type}
\end{figure}

While a timestamp can be directly used as it is, an index first needs to aggregate the total number of modifications per placement and 
calculate the comparison-index based on the deviation from the master.
Albeit, that this is not a very costly operation, it will have a significant impact when executed multiple times.
To obtain comparable results, the benchmarks have been executed with the most liberal degree of freshness per type, to avoid any side effects or fallbacks to the primary nodes.\\
Figure \ref{fig:eval_type} therefore presents the overall latency of a mixed workload accompanied by the different evaluation types.
As already suggested all types provide a very similar response time. A difference becomes only really obvious when e.g. comparing a natively usable
timestamp to a relative time delay. While the first one can be applied directly the latter first
needs to extract the commit time of all possible placements and apply the deviation before it is able to filter.
This plot shows that although only marginally different the utilized filter will still slightly impact the overall performance.




%%%%%%%%%%%%%%%%%%%%%%%%%%%%%%%%%%%%%%%%%%%%%%%%%%%%%%%%%%%%%%%%%%

\subsubsection{Freshness-Aware Read Operations}

As with write-only operations also the freshness can directly indicate a suitable store combination.
While write-operations focus on a trade-off between latency of the primary transaction and a faster convergence time,
the read-operations will consequently focus on the target where the read is applied. While regular read-operations exclusively target
primary placements, freshness-queries will mainly be redirected towards possibly outdated secondary replicas.


As described before, since freshness-related results are essentially influenced by the entity composition, we need an independent measurement.
Therefore we will compare two equal stores (HSQLDB) to again allow a base comparison and observe how the freshness specification itself impacts the results.


\begin{figure}[t] 
    \centering 
    \begin{tikzpicture}
        \begin{axis}[
            width  = 0.9*\textwidth,
            height=6cm,
            ymajorgrids = true,
            ylabel = {Avg. Response Time (ms)},
            xlabel = {Time (sec)},
            scaled y ticks = false,
            minor y tick num=4,
            minor x tick num=9,
            xmin=0,
            xmax=300,
            ymin=0,
            ymax=2500,
            grid,
            grid style={dotted},
            major tick style={very thick},
            legend style={
                        at={(1,1.05)},
                        anchor=south east,
                        column sep=1ex
                },
            legend cell align=left,
        ]
            
    
            \addplot[style={bblue,mark=*}, mark options={scale=0.5}] 
            table [x=time, y=100]{Plot/fresh0.dat};

            \addplot[style={rred,mark=x}, mark options={scale=0.5}] %[thick]
                table [x=time, y=50]{Plot/fresh0.dat};      
            
            \addplot[style={ppurple,mark=diamond*}, mark options={scale=0.5}]% [thin]
                table [x=time, y=both]{Plot/fresh0.dat};  
    
            \legend{  
                100\% Freshness, 
                50\% Freshness,
                0\% Freshness   
                }
        \end{axis}
    \end{tikzpicture}
    \caption{Response Time Comparision on Varying Proportions of Freshness.}
    \label{fig:fresh0}
\end{figure}



Therefore Figure \ref{fig:fresh0} shows a comparison of different degrees of freshness used within a pure read-only environment.
These results generally show that the utilized freshness alone is already sufficient to provide better read results than if no freshness is used at all.
As before this is inherently caused by the targeted store of the query. While all queries without a freshness specification will directly contact a primary node,
hence needing to apply a lock, freshness queries can contact the secondary node without a lock. Although in general shared-locks can be easily applied for regular read-operations, they still need to be formally acquired impacting
the throughput. The most benefit can therefore be gained by combining the freshness with regular operations. As depicted above, these combinations provide by far 
the best results. Despite that regular reads still need to acquire a lock, these different query types will consequently also target different stores. 
This allows an efficient load distribution across the utilized stores, to harvest the benefits of a distributed environment increasing the overall throughput.




Even with a mixed workload without any freshness constraints we already observed that the store constellation does have an impact on the overall latency 
and influences the decision process. 
These effects can also be considered when evaluating a partial workload on our interleaved setup with HSQLDB and PostgreSQL.


\begin{figure}[t] 
    \centering 
    \begin{tikzpicture}
        \begin{axis}[
            width  = 0.9*\textwidth,
            height = 7cm,
            x=120pt,
            major x tick style = transparent,
            ybar=.2cm,
            bar width=30pt,
            ymajorgrids = true,
            ylabel = {Avg. Response Time (ms)},
            xlabel = {Freshness Degree},
            symbolic x coords={0\%,100\%},
            xtick = data,
            scaled y ticks = false,
            enlarge x limits=0.40,
            ymin=0,
            ymax=4000,
            minor y tick num=4,
            major tick style={thick},
            legend style={
                    at={(1,1.05)},
                    anchor=south east,
                    column sep=1ex
            },
            legend cell align=left,
        ]
            \addplot[style={bblue,fill =bblue,mark=none}]
                coordinates {(0\%, 2504.659922) (100\%,3050 )};

            \addplot[style={rred,fill=rred,mark=none}]
                coordinates {(0\%, 2620) (100\%,2856.9465) };

            \legend{HSQLDB (Eager) - PostgreSQL (Lazy),
                PostgreSQL (Eager) - HSQLDB (Lazy) 
            }
        \end{axis}
    \end{tikzpicture}
    \caption{Impact of Store Selection based on Varying Proportions of Freshness -- Partial Freshness.}
    \label{fig:mixed}
\end{figure}

The illustration in \ref{fig:mixed}, therefore indicates the impact a given store constellation has on the overall throughput.
While regular operations perform better in the configuration targeting HSQLDB as the primary node,
whereas queries with freshness, target the secondary node essentially flipping the roles.
This is again caused by the target of the select statements. 
Since we are rather in a read-heavy environment the selection has quite the impact on the final result.
This proves that the store constellation will strongly influence the systems behavior as well as the designated workloads to use this constellation on.


%%%%%%%%%%%%%%%%%%%%%%%%%%%%%%%%%%%%%%%%%%%%%%%%%%%%%%%%%%%%%%%%%%
\subsubsection{Compound Operations} 

Now with every building block evaluated independently, we will combine and assess the given functionality jointly.


%%%%%%%%%%%%%%%%%%%%%%%%%%%%%%%%%%%%%%%%%
% Mixed with Freshness comparison
%%%%%%%%%%%%%%%%%%%%%%%%%%%%%%%%%%%%%%%%%

Based on the elementary comparison in Figure \ref{fig:write_decomposition} we have already established the heavy impact the capture-queue as well as the 
replication have on a single write-operation.

As mentioned in the implementation, depending on the current load on the system, we might observe contentions since the replication is done in parallel.
Furthermore, we might even come across situations where an administrator will directly define a placement as \emph{INFINITELY OUTDATED} to not receive any more updates
and retain its current version.\\
For the next benchmark, we therefore, want to see how the workload behaves if the replication is done in parallel if the replication distribution is suspended
and if the entire queue-capture is suspended.
This is again done using two HSQLDB stores, to reduce possible deviations across stores. 
Additionally, this benchmark introduces freshness queries, to observe how it influences the overall response time even when there is no active replication. 
The freshness evaluation utilizes a freshness-index of $0.7$ to filter and identify possible candidates based on the number of received modifications.


\begin{figure}[t] 
    \centering 
    \begin{tikzpicture}
        \begin{axis}[
            width  = 0.9*\textwidth,
            height=6cm,
            ymajorgrids = true,
            ylabel = {Avg. Response Time (ms)},
            xlabel = {Time (sec)},
            scaled y ticks = false,
            minor y tick num=4,
            minor x tick num=9,
            xmin=0,
            xmax=300,
            ymin=0,
            ymax=7000,
            grid,
            grid style={dotted},
            major tick style={very thick},
            legend style={
                        at={(1,1.05)},
                        anchor=south east,
                        column sep=1ex
                },
            legend cell align=left,
        ]
            
    
            \addplot[style={bblue,mark=*}, mark options={scale=0.5}] 
            table [x=time, y=active]{Plot/replication_impact.dat};

            \addplot[style={rred,mark=x}, mark options={scale=0.5}] %[thick]
                table [x=time, y=replication]{Plot/replication_impact.dat};      
            
            \addplot[style={ppurple,mark=diamond*}, mark options={scale=0.5}]% [thin]
                table [x=time, y=capture]{Plot/replication_impact.dat};  
    
            \legend{  
                Replication Active,
                Replication Suspended,
                Capture + Replication Suspended,
                }
        \end{axis}
    \end{tikzpicture}
    \caption{Impact of Replication Queue on Response Time using Freshness (Index=0.7) -- Partial Freshness.}
    \label{fig:replication_impact}
\end{figure}

As provided by Figure \ref{fig:replication_impact} we can again observe that the overall response time improves if there is no active replication running in the background.
Furthermore, the system performs best if it does not need to capture any changes at all, 
which essentially corresponds to the observations we have described in Figure \ref{fig:write_decomposition}.
Although this is rather a corner case, the replication suspension on the other hand is more likely to occur and still provide good response times.

Interestingly, we further observe peaks on the active replication graph. While the graphs without replication and capture mechanisms are more narrow,
the active replication queue oscillates quite heavily, negatively  impacting the average response time.\\
Further, we observe that these peaks start occurring after roughly one minute.
Since we used a modification deviation for our freshness specification, the system can quickly identify suitable placements at the beginning of the benchmark.
However, after some time the versions start deviating quite heavily from their eager version.
Since the updates are only propagated operation-wise, the replications cannot catch up with the primary operations anymore as already presented in Figure \refname{fig:converge24}.
Hence the system is not able to fulfill the tolerated freshness-level and
causes the algorithm to fall back to its primary placement, again investing additional time to combine and select a suitable combination of placements, which ultimately causes the peaks.




%%%%%%%%%%%%%%%%%%%%%%%%%%%%%%%%%%%%%%%%%
% Partition
%%%%%%%%%%%%%%%%%%%%%%%%%%%%%%%%%%%%%%%%%
\subsubsection{Partitioning} 

Our freshness and replication implementation essentially revolves around the handling of individual Partition Placements.
However, so far all tests only really considered unpartitioned entities to visualize basic differences. 

\begin{figure}[t] 
    \centering 
    \begin{tikzpicture}
    \begin{axis}[
        width  = 0.8*\textwidth,
        height=6cm,
        x=80pt,
        major x tick style = transparent,
        ybar=.2cm,
        bar width=25pt,
        ymajorgrids = true,
        ylabel = {Avg. Response Time (ms)},
        xlabel = {Partition Constellations},
        symbolic x coords={1},
        xtick = data,
        xticklabels={},
        scaled y ticks = false,
        enlarge x limits=0.40,
        ymin=0,
        ymax=3000,
        minor y tick num=4,
        major tick style={thick},
        legend pos=outer north east,
        legend cell align=left,
    ]
        \addplot[style={bblue,fill =bblue,mark=none}]
            coordinates {(1, 444.19055)};

        \addplot[style={rred,fill=rred,mark=none}]
            coordinates {(1, 364)};

        \addplot[style={ggreen,fill =ggreen,mark=none}]
            coordinates {(1, 2395)};

        \legend{ 
            5 Stores - 4 Partitions,
            2 Stores - 4 Partitions,
            2 Stores - Unpartitioned
            }
    \end{axis}
\end{tikzpicture}
    \caption{Impact of Partitioning on Freshness (Index=0.7), Across Lazy Replicated Stores -- Partial Freshness.}
    \label{fig:partition_result}
\end{figure}

Accompanied by the introduced locking changes and the promising results described in Figure \ref{fig:lock_comp}, we will evaluate the Partial Freshness workload in 
a partitioned environment. For one we will partition the table into four horizontal partitions. 
For one scenario we place all partitions on the eager node and then each partition respectively on four lazy replicated stores to achieve a highly distributed setup as before.
The second scenario will place all four partitions across two stores, where exactly one is eagerly and the other lazily replicated. 
These two variations are ultimately compared against an entire unpartitioned setup across two HSQLDB stores which is visualized in Figure \ref{fig:partition_result}.


Analogously, to the results obtained from the locking mechanism benchmarks, 
these results consequently indicate, that we achieve the best performance
if we indeed partition the entity and place all partitions on each entity across both stores. This natively allows the system to have the most possible 
flexibility in locking and accessing the individual data fragments. 
Ultimately this does not only mimic the performance gains we have already observed due to the locking 
but also extends it even further when ingesting freshness constraints.\\





%%%%%%%%%%%%%%%%%%%%%%%%%
\section{Workload Comparison}

Concluding with the evaluations and as illustrated with the last examples and benchmarks, the results of freshness constraints, 
inherently differs through the utilized workload.
On the basis of fully freshness optimized read-operations using an index of $0.7$, and two HSQLDB stores,
Figure \ref{fig:workload_comp} summarizes the impact of the individual workload classes on freshness-aware data management in general. 


As we have identified before the most extreme cases like read-only and write-only workloads achieve the best results,
since there are no negatively  impacting concurrent operations.
However, because these are not considered to be common workloads, the next best operation is considered to be a write-heavy workload 
which only partially focuses on reads.
This essentially mimics a highly transactional workload using the reads mainly for analytical purposes.


This shows that freshness-aware data management can indeed improve mixed workloads during write-heavy situations.

\begin{figure}[t] 
    \centering 
    \begin{tikzpicture}
        \begin{axis}[
            width  = 0.9*\textwidth,
            height = 6cm,
            ymajorgrids = true,
            ylabel = {Throughput (requests/sec)},
            xlabel = {Workload (R/W)\%},
            symbolic x coords={(100/0),(80/20),(60/40),(50/50),(40/60),(20/80),(0/100)},
            xtick = data,
            scaled y ticks = false,
            enlarge x limits=0.1,
            minor y tick num=3,
            ymin=0,
            ymax=60,
            major tick style={very thick},
            legend pos=outer north east,
            legend cell align=left,
            smooth
        ]
            \addplot[style={bblue,mark=*}] [thick]
                coordinates {
                    ((100/0), 54.7044553) 
                    ((80/20),16) 
                    ((60/40),10.4) 
                    ((50/50),12.52163117) 
                    ((40/60), 15.56761) 
                    ((20/80),16.99402276) 
                    ((0/100),19.05664)
                    };

        \end{axis}
    \end{tikzpicture}
    \caption{Workload Comparison of Freshness Related Queries (Index=0.7).}
    \label{fig:workload_comp}
\end{figure}


%%%%%%%%%%%%%%%%%%%%%%%%%%%%%%%%%%%%%%%%%%%%%%%%%%%%%%%%%%%%%%%%%%
%%%%%%%%%%%%%%%%%%%%%%%%%%%%%%%%%%%%%%%%%%%%%%%%%%%%%%%%%%%%%%%%%%
%%%%%%%%%%%%%%%%%%%%%%%%%%%%%%%%%%%%%%%%%%%%%%%%%%%%%%%%%%%%%%%%%%





\section{Discussion}
\label{sec:discussion}

The results generally show that although the replication will slightly mitigate the overall performance of the system, it still can correctly fulfill the 
requirements without largely interfering with the underlying systems. 
Despite that the operation-wise execution will gradually progress each placement as intended, the capture-queue inherently impacts the performance of parallel workload
when using a lazy replicated approach.

While eagerly replicated setups generally performed better in write-heavy environments, 
the compound results still show that freshness related queries are indeed suitable candidates when not focusing on a pure mixed-workload with equal amounts of 
read- and write-operations. 
Especially the partitioned cases showed promising results to optimize routing decisions to finally improve the overall performance.


Although the freshness queries already provided good results, they still suffered from a rather static freshness specification. 
Therefore it would make sense to extend and adapt the utilized benchmarks to adaptively adjust the degree of freshness during runtime
as needed and still providing reproducible results.

So far we only focused on the basic functionalities and suitable fields of application.
Since also the respective freshness degree will impact the choice of which of the proposed candidates fulfills the requirements of the query,
it would make sense to further benchmark different varying freshness constraints to observe their long-term behavior. 


%%%%%%%%%%%%%%%%%%%%%%%%%%%%%%%%%%%%%%%%%%%%%%%%%%%%%%%%%%%%%%%%%%




