% !TEX root = ../Thesis.tex
\chapter{Conclusion}
\label{c:conclusion}


With this implementation Polypheny-DB now provides functionalities to adjust itself to the concepts revolving around \emph{CAP} and \emph{PACELC} described in \ref{sec:cap}.
To let users choose between consistency and availability by decoupling primary and secondary updates and deferring refresh operations to a later point in time.
Due to this asynchrony it now efficiently supports hybrid workload. 


With this implementation we have introduced a possibility to allow system administrators or operators in general to define their replication requirements as needed. 
With the introduced replication strategies and states we can define on a table-level basis how updates are propagated within the system and can therefore directly influence 
the availability and consistency per object.\\
This immediately enables us to use possibly outdated nodes containing stale data to be used during retrieval to support analytical queries even in the presence of transactional load.
This does not only improve parallel processing but also allows the efficient usage of all available stores.





When partial replication is used, several of the underlying stores may qualify
for the execution of these queries. In order to avoid that single stores are overloaded, query processing and optimization
can effectively consider version selection and load balance the access among relevant replicas.

Although this work has shown in \ref{sec:benchmark} that is greatly improves the throughput of this inherently distributed system, the usage should still be considered with care.
Despite that we established certain counter measures 


At the end this work introduced several nuances of freshness to support varying use cases and requirements.
Albeit not being able to support Serializable Snapshot Isolation, the implementation still offers 

Since we are certain, that we do not have infinitely available versions as in conventional multi-version database systems.
Even the freshness specification without any determined tolerated level, promises a certain level of freshness by design.


%%%%%%%%%%%%%%%%%%%%%%%%%%%%%%%%%%%%%%%%%%%%%%%%%%%%%%%%%%%%%%%%%%

\section{Outlook}

\subsection{Tuneable Consistency}
The introduced implementation sketched in section \ref{c:implementation} reduces the overall consistency of the primary transaction,
to improve the overall response time of the system.\\ 
But since this trade-off between availability and consistency certainly depends on the use case or service requirements, it would be beneficial.
Hence, an extension to the described model could easily allow to adjust the required consistency as needed. 
This could be either done by the mentioned usage of policies, described in section \ref{sec:polcies} or with.

Instead of labeling fixed data placements to receive updates eagerly, we could allow a more flexible approach that ts is sufficient if already placement 
shall receive the update, disregarding its role. The predefined replication state can be therefore omitted. 
Such approaches can then be easily combined with tuneable consistency to allow self adjusting data placements  adapting to individual use cases.

\subsection{Locking}
Reduce locking to a physical partition level  (partition placement)




%%%%%%%%%%%%%%%%%%%%%%%%%%%%%%%%%%%%%%%%%%%%%%%%%%%%%%%%%%%%%%%%%%

\tocless\section{Global Replication Strategies}
This implementation has only introduced the specification of table-level entities like entire data placements to be defined as eagerly or lazily replicated.
Although this introduces a high degree of flexibility, it still might be desirable to define certain policies that entire schemas or even databases automatically 
receive a lazy replication, while still ensuring the overall placement constraints.\\
This concept could be intended even further by applying it to a distributed setup of Polypheny, that replicates data autonomously to certain regions based on the given 
This extension could leverage the introduced freshness-awareness to consider off-site locations for even more parallel workload.


%%%%%%%%%%%%%%%%%%%%%%%%%%%%%%%%%%%%%%%%%%%%%%%%%%%%%%%%%%%%%%%%%%

\subsection{Policies}
\label{sec:polcies}

According to the idea, to generally relax consistency or allow a fine grained way of letting data owners decide what kind of consistency shall be desired for their object. 
Since freshness can be considered a trade-off between availability and consistency it is only fair to let users decide which level of consistency to enforce and 
how the freshness should be handled. We therefore propose the notion of policies to guide the system.
Policies are essentially intentions and desired states how the system should behave in various situations. 
The system can apply them when manual or automatic system maintenance is performed.
They shall therefore be introduced for any kind of configurable behaviour to allow any custom tailored behaviour. \\

Policies can be inherented and applied to any kind of object. When applied to a schema all entities inherent that polcies.
However, a different kind of policy with the same type can be applied to an entity overriding the one inherented by the parent.
Policies are about background processing how metadata and constraints on different objects are enforced. 
They provide a lightweight version of UDFs (User-Defined-Functions) to build custom-tailored behaviours into the system.
Other than the central configuration which is used to define core system behaviours. 
Such policies can be defined as:\\


\textbf{Consistency Policies}\\
Provide a notion of tuneable consistency, where users should be able to decide which levels to fulfill.
In such a case an administrator could choose to define how many primary replicas an object should always contain.
This would directly impact the constraints on the table restricting users to remove more placements than defined by that policy.\\


\textbf{Freshness Policies}\\
Can be utilized to define behaviour on freshness related actions. As \cite{fekete:2018} stated it is crucial to define how far a replica can diverge from the true
and up-to-date value before a refresh transaction has to be critically executed.
Additionally, we could use these types of policies to let object owners define how their data shall be identified and consequently updates are propagated.
In such a way the system would stay customizable and would allow any methods discussed in \ref{stale} to be used dependent on the use case.
Considering how refreshes are triggered one policy for example might has chosen to defer a propagation transaction entirely hence it won't try again and
essentially waits for another chance when a new update-transaction is being executed and will trigger the service again. Another policy could suit other use cases better 
and assist by constantly querying the storage's performance metrics to decide if an update should be executed.\\

Due to the inherent heterogenous nature of the polystore systems itself, use cases may widely vary. Hence, in gerneal there is no need to impose a 
general notion of freshness that is valid for all applications. Some might consider using CDC, while other prefer a partition approach or materialized views.
However, with policies these can all be implemented.\\


%\subsection{Policy usage}

Since applications that are being served by polystores are very different to each other, they might have different requirements. 
Therefore, different object types shall be supported.

A policy shall generally be created inside a database.
\begin{verbatim}
CREATE POLICY policy_name as <configuration>;
\end{verbatim}

These policies can then be added to any object.
\begin{verbatim}
ALTER TABLE dummy ADD POLICY policy_name;
\end{verbatim}

Users should always be able to list information on all applied policies on any object.
\begin{verbatim}
SHOW POLICIES ON (DATABASE | SCHEMA | TABLE ) object_name;
\end{verbatim}


Furthermore they should be able to view the content of any policy: 
\begin{verbatim}
DESCRIBE POLICIY policy_name ON (DATABASE | SCHEMA | TABLE ) object_name; 
\end{verbatim}

To complement the idea of policies we also need a central \emph{Policy Manager} which ensures that the specified intentions are ensured.
The manager shall be added as an additional central component and acts as a verification layer whenever meta information on objects will change.

