% !TEX root = ../Thesis.tex
\chapter{Conclusion}
\label{c:conclusion}


With this implementation Polypheny-DB now provides functionalities to adjsut itself to the concepts revolving around \emph{CAP} and \emph{PACELC} described in \ref{sec:cap}.
To let users choose between consistency and availability by decoupling primary and secondary updates and deferring refresh operations to a later point in time.
Due to this asynchronicity it now efficiently supports hybrid workload. 

%%%%%%%%%%%%%%%%%%%%%%%%%%%%%%%%%%%%%%%%%%%%%%%%%%%%%%%%%%%%%%%%%%

\section{Outlook}

\subsection{Tuneable Consistency}
The introduced implementation sketched in section \ref{c:implementation} reduces the overall consistency of the primary transaction,
to improve the overall response time of the system.\\ 
But since this trade-off between availability and consistency certainly depends on the use case or service requirements, it would be beneficial.
Hence, an extension to the described model could easily allow to adjust the required consistency as needed. 
This could be either done by the mentioned usage of policies, described in section \ref{sec:polcies} or with.

\subsection{Locking}
Reduce locking to a physical partition level  (partition placement)


%%%%%%%%%%%%%%%%%%%%%%%%%%%%%%%%%%%%%%%%%%%%%%%%%%%%%%%%%%%%%%%%%%

\tocless\section{Unified transaction model for semantically rich operations}
Introduce semantically rich operations inside polypheny? 
Maybe not on the scope of an application level but to encompass the underlying transactions