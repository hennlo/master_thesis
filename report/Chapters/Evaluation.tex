% !TEX root = ../Thesis.tex
\chapter{Evaluation}
\label{c:evaluation}

This chapter is separated into 

\section{Goal}
The evaluation has two goals the correctness as well as the impact of data freshness onto different kinds of workloads.

Verify and validate the correctness as well as the completeness of the implementation based on several characteristics.
These include the correct execution of lazy replication, the possibility to refresh statements on demand.

Impact of the replication engine on the underlying performance, if freshness indeed increases the overall parallel writes on the system.
Or if it is just marginally lower than before. Also compare this to the overall introduced overhead. And if the change was wort it

\section{Correctness}

\section{Benchmarks}

\todo{Explain why it is necessary to verify the solution with different combinations}

\subsection{Evaluation Environment}
\todo{Ellaborate and throughly explain why the environment was chosen and how the test was executed for the sake of repoducibility}


\subsection{Evaluation Procedure}
The following steps outline the procedure for benchmarking data freshness within Polypheny-DB.

\todo{ Execute benchmarks on multimodal dbs. as well as different kind of }

\todo{ Talk about implementation of freshness characteristic in many query languages}



\section{Discussion}
\label{sec:discussion}
The result generally show