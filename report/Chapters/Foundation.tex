% !TEX root = ../Thesis.tex
\chapter{Foundation}
\label{c:Foundation}

This chapter describes concepts and general foundations, which are necessary to supplement 
the content of this thesis. 

\section{Polystores}

\subsection{Polypheny-DB}

\section{Data Partitioning}

\subsection{Vertical Partitioning}
\subsection{Horizontal Partitioning}

\section{Temperature-aware data management}

\section{Data Replication}
In distributed setups...

\subsection{Eager Replication}
\todoMissing{Insert}

\subsection{Lazy Replication}
Automatically results in \emph{Eventual Consistency}. Lazy replication therefore already have the characteristic of outdated nodes that leave several versions behind.
\todoMissing{Insert}


\section{CAP Theorem}

Although CAP was essentially introduced to support primarly the differentiation between \emph{Availability} and \emph{Consistency} it only considers the failure scenario.
Therefore, an extension was introduced for the non failure case.
PACELC (Else Latency or Consistency). Referring to choose between Latency hence availability or consistency. 
Talk about the extension to PACELC maybe even in subsection

