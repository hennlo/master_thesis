% !TEX root = ../Thesis.tex
\chapter{Foundation}
\label{c:Foundation}

This chapter describes concepts and general foundations, which are necessary to supplement 
the content of this thesis. These foundations are mainly associated with the topics of distributed data management.
\todoMissing{Based on the topics of  Distributed Data Management}

\section{Polystores}

The decision which data structure to use and built upon is a crucial step for the overall performance of a systems design, as suggested by
H.Plattner and B.Leukert\cite{plattner2015}.\\
While row-oriented data stores might be useful and preferred for write-heavy transactional 
workloads they are rather insufficient for purely analytical workload which would rather benefit from a
column-oriented data store with less write operations\cite{sigmond2008}.\\

Despite the fact that nowadays there exist a variety of Database Management Systems (DBMS) which were originally created with an intention to
support specific scenarios,
applications are getting more complex relying on various requirements and characteristics 
to serve multiple use cases at once.
That is why modern day applications can not solely rely on one storage technology alone. 
Consequently Multi- and Polystore systems have emerged. \\
\\
While multistore database systems aim to combine and manage data across heterogeneous data stores,
polystore systems are essentially based on the idea of combining multistores with
\textit{polyglot persistence}\cite{polypheny2020}.
Polyglot persistence is a term which refers to a practice originated from the concept 
of \textit{polyglot programming} or microservice architectures, to utilize different 
programming languages for different task requirements following a best-fit approach\cite{fowler2011}. \\
Along this paradigm, polystores want to utilize multiple data storage technologies to
fulfill different needs for different application components in order to cope
with mixed and varying workloads.

\subsection{PolyDBMS}


\section{Data Partitioning}
\label{sec:part}

Beside the utilization of several data storage engines, the data model and structure 
will have an enormous impact on the overall performance. Depending on the query or 
how data is accessed ,data partitioning can be used to increase the efficiency and 
maintainability of the system\cite{Agrawal_2004}.\\
The process of data partitioning refers to splitting the data into logically and sometimes even
physically seperated fragments.

In general data partitioning can be distinguished between two variations.

\begin{description}
    \item [Vertical Partitioning] is usually applied during the design of a data model inside a 
    database. This involves the creation of tables with fewer columns and therefore using additional 
    tables to store the remainder of columns.\cite{vertical_1984}. This approach is often used in the 
    context of Normalization of a data model\cite{normalization_2012}. 
    \item [Horizontal Partitioning] refers to the partitioning of objects like tables 
    into a disjoint set of rows that can be stored and accessed separetely\cite{horizontal_1982}.
    To support this explicit form of partitioning there exist several partition algorithms.
    The most common ones are List, Range and Hash partitioning. These algorithms can be applied to a
    table based on an arbitary column which results in a fragmentation of the table 
    based on the data values of the selected column.
\end{description}

Data partitioning generally enables a system to process data concurrently and 
to some extent even in parallel. Considering that access to data can be 
efficiently load balanced and therefore enhances the throughput per query.\\

Altough data partitioning is often associated with the improvement of query performance.
It can be also be used to simplify the operating of a DBMS cluster and therefore help 
to increase the overall availability.
Through the replication of partition fragments, the data resilience of the system
can be improved. Even if part of the data storage nodes are temporarily not 
reachable, your system still might be fully operational and available due to the 
replication and distribution of data fragments, which is still one of the main 
pursuits of current cloud providers\cite{dbre2017}.

\subsection{Vertical Partitioning}
\subsection{Horizontal Partitioning}

\todo{Or rather an entire chapter with Properties of Distributed Systems/ Data Management}
\section{Temperature-aware Data Management}

\section{Data Replication}
In distributed setups...
Cloud providers often tend to use data partitioning to improve the query performance
by replicating certain partitions of data to where it is actually needed\cite{cloudpart_2012}.\\

\subsection{Eager Replication}
\todoMissing{Insert}

\subsection{Lazy Replication}
Automatically results in \emph{Eventual Consistency}. Lazy replication therefore already have the characteristic of outdated nodes that leave several versions behind.
\todoMissing{Insert}

\section{Concurrency Coontrol}
\subsection{(SS)2PL}
\subsection{MVCC}
\subsection{Discussion}

\section{CAP Theorem}

Although CAP was essentially introduced to support primarly the differentiation between \emph{Availability} and \emph{Consistency} it only considers the failure scenario.
Therefore, an extension was introduced for the non failure case.
PACELC (Else Latency or Consistency). Referring to choose between Latency hence availability or consistency. 
Talk about the extension to PACELC maybe even in subsection

