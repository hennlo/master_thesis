% !TEX root = ../Thesis.tex
\chapter{Motivational Scenario}
\label{c:motivation}

Explain scenario and why especially necessary on polystore systems
transactional UPTODATE workload on postgres and 
Analytical in in-memory order column-oriented DB. 

(Imagine a distributed system)

COnsider locking in a distributed setup. Why this is limited by the slwoest performing node.
(maybe cite) Amdahls law .

\todo{This might not be suitable in Motivational Scenario}
We can now harvest the benefits of a polystore system with their distinct replication engines

Imagine an \emph{Enterprise Resource Planning} (ERP) system, containing several different customers.
This comprehensive system is able to provide purchasing, accounting, inventory stocking and warehouse management within one system.
Since these systems are inherently used to reproduce an entire supply chain, they can be easily be used to report on different sales metrics
or run analytical queries to aggregate certain multidimensional results. This is certainly desirable to analyze the current state of the company as well as adequately adapt 
to new trends or opportunities. However, even in we imagine that we have a distributed system \\
If our system however only supports strong consistency by eagerly replicating all incoming updates to every available node, 
we are limited by the slowest performing node in this setup. During these write operations no could

Explain that this system in this scenario does not allow concurrent writes and reads. Since reads do not alter that data, we can provide multiple 




However, since ERP \todoMissing{Cite this} are strong transactional systems receiving write-heavy workload the write speed is essentially bound by the lowest performing

This does not only limit locking situations but also increases the databases' response time. Furthermore, this enables this system even in a distributed setup to work 
with several nodes. Even if not all nodes are directly notifying the system that they have succesfully finished the operation we could continue with the operation.
Depending on the tolerated level of consistency we give in on consistency to improve availability. This can of course also be tuned to wait until a majority or defined number of nodes 
have comited the transaction.