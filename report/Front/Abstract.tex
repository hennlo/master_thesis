% !TEX root = ../Thesis.tex
\chapter{Abstract}

In recent years data has not only grown in volume and variety but also in importance.
It has developed into an essential driver to advance industrial as well as scientific achievements.
This influence elevated the need to store data reliably in different distributed locations,
to ensure its availability even in the presence of partial failures.

But, since the cost of maintaining several replicas per data object
is rather high, systems often need to compromise between access times and consistency.
Therefore they usually limit the number of synchronously updated replicas, and lazily replicate
the changes to the remaining nodes in a cluster.
To cope with constantly varying requirements and the increased complexity of systems,
efficient and cost-optimized data processing has become a crucial operational aspect 
to allow access to distributed data.

Consequently, to efficiently utilize all resources in a system, 
various notions of freshness have been established, to allow a more flexible approach to
access distributed data.
These provide the possibility to specify freshness demands and define whether
the most recent data is required or if a request can be satisfied with slightly outdated data as well.\\


In this thesis, we summarize and define the fundamental features of freshness-awareness and exemplify the 
implementation on the basis of the Polystore system, Polypheny-DB.

We, therefore, establish a lazy replication algorithm to defer updates on selected replicas
and propagate changes across the underlying stores, to create multiple versions of 
data objects.
Accompanied with individual query extensions and various freshness metrics, we allow to centrally process 
these possibly outdated objects and leverage the key benefits of the encompassed stores,
to increase the overall efficiency of the system.\\

We will evaluate the individual functional building blocks, necessary to establish freshness-awareness,
to identify the impact the implementation has on the system. Further, we want to observe how the
key functionalities of the underlying stores, influence the behavior of the solution.

The results show that although the solution is not 
universally applicable to every scenario, it significantly improves 
the overall execution, especially in distributed Polystore systems.